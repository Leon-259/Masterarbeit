\section{Forschungsfrage und Zielsetzung der Arbeit} \label{sec:Zielsetzung}

Die zentrale Forschungsfrage dieser Arbeit lautet: 

\begin{quote}
    \textit{Wie kann unter Anwendung von Methoden des maschinellen Lernens ein System entwickelt werden, das die Detektion unerwünschten Verhaltens ermöglicht?}
\end{quote}

Durch die Frage wird die Relevanz der unerwünschten Verhaltensweisen verdeutlicht. Dabei handelt es sich um die Verhaltensweisen, bei denen wirtschaftliche Verluste entstehen und bei denen Tiere zu Schaden kommen. 
Im Fokus dieser Arbeit stehen Kämpfe und Kontrollgänge. Explizit wird maschinelles Lernen genannt. Dadurch wird deutlich, dass der Fokus auf moderne Technologien gelegt wird und dass ein Ansatz für die Präzisionstierhaltung verfolgt wird. \par

Aus dieser Fragestellung und dem Hintergrund lassen sich Ziele ableiten, welche in dieser Arbeit verfolgt werden sollen. Hauptziel ist die Realisierung einer Verhaltensklassifikation. Diese soll zuverlässig das Verhalten im Mastputenstall erkennen können. Direkt aus der Fragestellung geht hervor, dass dazu maschinelle Lernmethoden verwendet werden sollen. Relevant ist dabei vor allem die Anwendung in der industriellen Geflügelhaltung. Aus diesem Grund soll die Verhaltenserkennung in Echtzeit erfolgen. Wie in \autoref{sec:Hintergrund} dargestellt ist dies elementar, um Einfluss auf das Tierverhalten zu nehmen. Eine solche Einflussnahme kann im Rahmen eines umfangreichen Stallmanagmentsystems erfolgen. Um die Verhaltenserkennung für solche übergeordneten Systeme nutzen zu können, wird sie als Modul entwickelt. Dadurch ist sie leicht integrierbar. Dafür muss das Modul das Verhalten automatisiert erkennen können.\par

Zusammengefasst lassen sich folgende Ziele für die Arbeit formulieren.

\begin{itemize}
    \item \textbf{Modularität}: \\
    Es soll ein Modul entwickelt werden. 
    \item \textbf{Echtzeitfähigkeit}: \\
    Die Datenverarbeitung soll in Echtzeit möglich sein.
    \item \textbf{Verhaltenserkennung}: \\
    Das Verhalten der Mastputen muss korrekt erkannt werden.
    \item \textbf{Verwendung von maschinellem Lernen}: \\
    Methoden des maschinellen Lernens sind anzuwenden.
    \item \textbf{Zuverlässigkeit}: \\
    Die Verarbeitung soll zuverlässig und konstant laufen.
    \item \textbf{Automation}: \\
    Das Modul soll Daten automatisiert verarbeiten. 
\end{itemize}
