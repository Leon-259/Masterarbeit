\section{Zielsetzung der Arbeit} \label{sec:Zielsetzung}
Ziel der Arbeit ist es, ein echtzeitfähiges Modul zu entwickeln. Dieses soll automatisch Verhaltensweisen erkennen können. Damit das Modul in einer Anwendung genutzt werden kann, muss die Verarbeitung zuverlässig laufen. Diese Ziele sind hier einmal übersichtlich dargestellt.


\begin{itemize}
    \item \textbf{Modularität}: \\
    Es soll ein Modul entwickelt werden. 
    \item \textbf{Echtzeitfähigkeit}: \\
    Die Datenverarbeitung soll in Echtzeit möglich sein.
    \item \textbf{Verhaltenserkennung}: \\
    Das Verhalten der Mastputen muss korrekt erkannt werden.
    \item \textbf{Verwendung von maschinellem Lernen}: \\
    Methoden des maschinellen Lernens sind anzuwenden.
    \item \textbf{Zuverlässigkeit}: \\
    Die Verarbeitung soll zuverlässig und konstant laufen.
    \item \textbf{Automation}: \\
    Das Modul soll Daten automatisiert verarbeiten. 
\end{itemize}
