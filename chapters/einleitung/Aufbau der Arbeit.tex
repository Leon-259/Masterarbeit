\section{Aufbau der Arbeit}
Diese Arbeit befasst sich mit der Konzeptionierung, der Umsetzung und der Evaluation eines Moduls für die automatische Klassifikation von Verhaltensweisen von Mastputen. Dazu findet zunächst eine Auseinandersetzung mit der Literatur zum Aufbau von Modellen des maschinellen Lernens statt. Solche Systeme benötigen Merkmale, mit denen sie Zusammenhänge in Daten zu einem Sachverhalt lernen können. Für die Erschaffung solcher Merkmale stehen vorhandene Daten und Systeme aus dem \textit{\acrshort{OptiLiMa}} Projekt zur Verfügung. Das sind Videos, die das Stallgeschehen aufgezeichnet haben, Detektionen der Puten auf Basis der Videos und ein System, welches aus den Detektionen Trajektorien generieren kann.\par

Da die Trajektorien die Einbeziehung von Individualverhalten in das System ermöglichen, ist zu überprüfen, wie korrekt die generierten Trajektorien das Geschehen abbilden. Dazu  findet eine Auseinandersetzung mit der Evaluation von Multi-Objekt Tracking Systemen (MOT) statt.\par

Im \autoref{chap:Theoretische Rahmen} werden die theoretischen Grundlagen dargestellt, die für ein Verständnis des Vorgehens notwendig sind. Es werden die Grundlagen von MOT (\autoref{sec:MOT Grundlagen}) und vom maschinellen Lernen (\autoref{sec:Grundlagen ML}) zusammengefasst. Im Detail wird auf Evaluationsmethoden von MOT Systemen eingegangen (\autoref{sec:MOT Metrics}) und zwei Algorithmen zum Generieren von Trajektorien werden präsentiert (\autoref{sec:Überblick Tracker}). Das Kapitel startet mit der Theorie zu MOT, da die Evaluation der Trajektorienqualität eine wichtige Voruntersuchung, im Aufbau des maschinellen Lernmodells war. Die dargestellte Theorie zum maschinellen Lernen beinhaltet den Machine Learning Workflow (\autoref{sec:MLWF}). Dieser beschreibt den Prozess, der beim Aufbau lernfähiger System durchlaufen wird. Die Ausführung bietet eine erste Orientierung zum Vorgehen in der Arbeit und wichtige Hintergründe zum Verständnis der Methodenentwicklung werden erläutert. \par

Das Vorgehen erfolgte nach dem Maschine Learning Workflow. Das geht auch aus dem \autoref{chap:Methodenentwicklung} hervor. Ein initiales Modulkonzept biete Orientierung für den weiteren Aufbau (\autoref{sec:Meth gesamtkonzept}). Das Vorgehen bei der Bewertung des MOT Systems wird erläutert (\autoref{sec:Meth MOT}) und anschließend werden die Prozessschritte des Maschine Learning Workflows durchlaufen. Das Kapitel schließt mit einem finalen Modulkonzept ab, wie auch mit einem Konzept zur Simulation der Anwendung (\autoref{sec:Meth FinalKonzept}). Um die Funktionalität des Moduls im Mastputenstall zu prüfen und um die Zielerreichung zu evaluieren, wird der Anwendungsfall simuliert. Ein Feldtest ist nicht möglich, da die Erfassungssysteme nicht mehr im Stall installiert sind. \par

Im \autoref{chap:Umsetzung} ist die Umsetzung der entwickelten Methoden und Konzepte dargestellt. \par

Einige Zwischenergebnisse sind wichtig für das Verständnis einiger Entscheidungen. Diese Ergebnisse und die daraus resultierenden Entscheidungen sind in der Methodenentwicklung an den entsprechenden Stellen zusammengefasst. Eine detaillierte Ausführung der Ergebnisse ist in \autoref{sec:Ergeb ModEntwi} zu finden. Hier findet die MOT Evaluation statt (\autoref{sec:Ergebnisse MOT}), wie auch die Auswertung der Simulation und die Bewertung des Moduls (\autoref{sec:Ergeb Sim}). \par

In \autoref{chap:Fazit} werden die Erkenntnisse und Ergebnisse dieser Arbeit zusammengefasst (\autoref{sec:Zusammenfassung}). Es wird auf die Beantwortung der Forschungsfrage und die gesetzten Ziele eingegangen (\autoref{sec:FrageundZiele!}). Ebenfalls findet ein Ausblick auf nicht umgesetzte Ideen, sowie weiterreichende Ansätze und Projekte statt (\autoref{sec:Ausblick}).

Die Arbeit ist in deutscher Sprache verfasst, nutzt jedoch in einigen Fällen etablierte englische Fachbegriffe, um die Klarheit zu gewährleisten.  \par

% https://www.repo.uni-hannover.de/bitstream/handle/123456789/14132/Dissertation_Plappert_genehmigt_TIB.pdf?sequence=1&isAllowed=y

