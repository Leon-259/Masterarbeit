\section{Aufbau der Arbeit}
\begin{itemize}
    \item wir beginnen mit den Theoretischen Grundlagen die Notwendig sind für das Verständnis der Arbeit, Anschließend wird erklärt welche Methoden entwickelnt wurden, um das Modul aufzubauen. Danch wird erläutert wie die Methoden angewendet wurden und wie die Praktische umsetzung des Moduls erfolgt ist. Darauf folgt die ausführliche darstellung und die Diskussion der zwischen Ergebnisse, sowie der finalen Evaluation des Moduls. Abschließend folgt ein Fazit 
    \item Die \acrshort{MOT} untersuchung wir in jedem kapitel an den anfang gestellt, da das \acrshort{ML} Modul stark auf dem Tracking aufbaut. Somit ist ein verständnis des Tracking-Systems Essentiell für das Verständnis der Arbeit
    \item Da es in der Arbeit nicht darum geht ein \acrshort{MOT}-System aufzubauen, sondern lediglich daraum zu überprüfen, ob die vorhandenen Optionen sich für das \acrshort{ML}-Modul eignen, wird sich darauf beschränkt die notwendigen Grundlagen zum verständnis der Evaluation zu erläutern. Weiterreichende theorie, ansätze und konzepte zum \acrshort{MOT} werden nicht bbetrachtet.
    \item Da die Detektionen zu beginn bereits vollständig abgeschlossen war, konnte in der Arbeit auf diese keinen einfluss mehr genommen werden. LEdiglich das ankünfende Assoziationssystem ist ließ sich beeinflussen, weshalb der hauptfokus der Untersuchungen auf der Assoziation liegt. Das Detetkionssystem wird ebenfalls untersucht, jedoch wird weniger fokus auf die Technischen Details dieses Systems gelegegt 
    \item Auch wenn diese Arbeit auf deutsch verfasst ist, so werden doch einige Englische begriffe verwendet, da sich diese in der Literatur durchgesetzt haben. Um hier eine umständliche eindeutschung zu vermeiden, welche dem Verständnis eher schadet als gut tut werden in diesen Fällen die konventionen der Literatur verwendet.
    \item Verwendete Abkürzungen werden einmal erwähnt und sind ansonsten im Abkürzungsverzeichnis zu finden 
    \item Fachbegriffe sind in einem Glossar am ende nochmal erläutert.
\end{itemize} 