\section{Aufbau der Arbeit}
Diese Arbeit befasst sich mit der Konzeptionierung, der Umsetzung und der Evaluation eines Moduls für die automatische Klassifikation von Verhaltensweisen von Mastputen. Dazu findet eine Auseinandersetzung mit der Evaluation von Multi-Objekt Tracking Systemen statt und mit Methoden und Prozessen im maschinellen Lernen. Die Arbeit teilt sich in den theoretischen Rahmen, die Methodenentwicklung, die Umsetzung und die Darstellung und Diskussion der Ergebnisse. Abschließen tut die Arbeit mit einem Fazit. \par

Im theoretischen Rahmen werden die theoretischen erläutert, die für das Verständnis der Arbeit notwendig sind. In der Methodenentwicklung sind die Konzepte und Vorgehensweisen dargestellt, welche für die Umsetzung entwickelt wurden. In der Umsetzung wird erklärt, wie die Konzepte realisiert wurden. In der Darstellung der Ergebnisse befinden sich die Auswertungen zu den Tests und Untersuchungen, welche beim Modellaufbau durchgeführt wurden. Auch die Diskussion der Ergebnisse ist in diesem Kapitel. Im Fazit werden die Erkenntnisse zusammengefasst. Die Erfüllung der Ziele und die Beantwortung der Forschungsfrage werden überprüft. Für Begriffserklärungen existiert ein Glossar.\par

Hauptfokus der Arbeit ist das Modul zur Verhaltensklassifikation. Zusätzlich wird sich intensiv mit der Bewertung von Multi-Object Tracking Systemen befasst. Die Qualität des zu überprüfenden Multi-Object Tracking Systems besitzt Relevanz für das Modulkonzept. Die Erkenntnisse der Evaluation beeinflussen das Konzept. Aus diesem Grund wird sich in den Kapiteln zunächst mit Multi-Object Tracking Systemen befasst und anschließend mit den Methoden des maschinellen Lernens.\par

Verfasst ist die Arbeit auf Deutsch. In einigen Fällen haben sich jedoch in der Literatur englische Bezeichnungen etabliert. Es wurde sich dafür entschieden, in solchen Fällen ebenfalls die englischen Begriffe zu verwenden. Die Verwendung von wenig etablierten Eindeutschungen wurde eher als hinderlich für das Verständnis empfunden. \par
