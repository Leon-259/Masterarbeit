\section{Hintergrund und Relevanz der Thematik}
Produkte tierischem Ursprungs, wie Milch, Eier und Fleisch sind wichtiger Bestandteil der menschlichen Ernährung. Der weltweite Fleischkonsum ist am Steigen. Gerade die Produktion von Geflügel hat in den letzten zwei Jahrzehnten ein enormes Wachstum erlebt \cite{StatistischesBundesamt.20200212}. Durch dieses wirtschaftliche Interesse und die zunehmende Digitalisierung findet zunehmend Informationstechnologie Einzug in die Tierhaltungsindustrie. Dies wird als Präzisionstierhaltung bezeichnet. In der Vergangenheit basierten die getroffenen Entscheidungen in der Tierhaltung vor allem auf Erfahrungswerten des Tierhalters. Die Präzisionstierhaltung zielt auf datenbasierte Entscheidungen ab. In diesem Zusammenhang wächst das Interesse am Einsatz von maschinellem Lernen in der Tierhaltung. \cite{Garcia.2020}.\par

Stark im Fokus der Forschung liegt dabei die Klassifikation von Tierverhalten. Bestrebungen zur Verhaltensanalyse mittels maschinellem Lernen existieren für die wichtigsten Nutztierarten. In \cite{VazquezDiosdado.2019} werden Lernalgorithmen eingesetzt, um das Verhalten von Schafen zu klassifizieren. In \cite{VazquezDiosdado.2015} wird eine Verhaltensklassifikation von Milchkühen vorgestellt. Auch die Anwendung bei Schweinen wird untersucht \cite{Tran.2023}. Viele Ansätze, die für größere Tiere erfolgreich anwendbar sind, sind unpraktikabel für Geflügel. Die hohe Tierdichte und große Ähnlichkeit sind herausfordernd. Dennoch wird auch in diesem Bereich geforscht \cite{Chen.2023, Gonzalez.2020, Nasirahmadi.2020}. Das Tracking von Tieren spielt dabei oftmals eine wichtige Rolle. Während in der Vergangenheit oft Sensoren am Körper die Bewegungen aufzeichneten, finden inzwischen Techniken des maschinellen Sehens verbreitet Einsatz.\par

Diese Arbeit befasst sich mit der Verhaltensklassifikation von Mastputen. Sie findet im Rahmen des Forschungsprojekts \textit{\acrshort{OptiLiMa}} (\textit{Optimierung des Lichtmanagements für die Haltung von Mastputen}) statt. Das Projekt wird gefördert von der \textit{landwirtschaftlichen Rentenbank}. Ziel von \textit{\acrshort{OptiLiMa}} ist die Untersuchung des Einflusses der Stallbeleuchtung auf Mastputen. In dieser Arbeit wird ein Modul entwickelt, welches automatisch das Verhalten im Mastputenstall klassifiziert. Es baut auf vorhandenen Systemen auf und soll modular in eine Verarbeitungskette integriert werden können. In Bezug auf das Forschungsprojekt ist eine automatische Klassifizierung von Verhaltensweise auf mehrere Arten und Weisen relevant. Es vereinfacht eine Auswertung der Auswirkung der Beleuchtung erheblich. Das gilt jedoch nicht nur für die Beleuchtung. Generell wird durch ein solches System die Untersuchung externer Einflüsse ermöglicht. Das modulare Konzept ermöglicht weiterreichende Anwendungen. Beispielsweise ist eine Integration in eine Stallregelung möglich, welche auf Basis des Verhaltens Maßnahmen ergreift.\par

Somit bietet ein solches Modul Potenzial für die Forschung, aber auch positive Auswirkungen auf wirtschaftliche Aspekte sind zu erzielen. Die Kontrolle des Stalls und der Mastputen erfolgt in der Praxis manuell durch Menschen. Ein Mensch muss den Stall betreten und einen Kontrollgang vollziehen. Durch eine automatische Überwachung des Stallgeschehens können Kontrollgänge reduziert werden, das spart Arbeitskraft ein. Auch das Erkennen von verletzten Tieren ist denkbar, sowie die Erkennung von Krankheiten. Dadurch ist Ausschuss zu reduzieren \cite{Chen.2023, Garcia.2020}. Auch liegen hier Vorteile für den Tierschutz. Eine automatische Überwachung und Regelung verspricht Bedingungen, die besser an die Tiere angepasst sind.\par

Es ist zu sehen, dass die Entwicklung eines Moduls zur automatischen Klassifikation von Verhaltensweisen von Mastputen Potenzial bietet für die Forschung, die Wirtschaft und den Tierschutz. Dabei ist das Modul ein wichtiger Schritt, um weiterreichende Anwendungen aufzubauen. 
