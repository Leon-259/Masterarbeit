\section{Hintergrund und Relevanz der Thematik} \label{sec:Hintergrund}
Produkte tierischem Ursprungs, wie Milch, Eier und Fleisch sind wichtiger Bestandteil der menschlichen Ernährung. Weltweit steigt der Fleischkonsum, wie auch die Fleischproduktion. Gerade die Produktion von Geflügel hat in den letzten zwei Jahrzehnten ein enormes Wachstum erlebt \cite{StatistischesBundesamt.20200212}. Dieses wirtschaftliche Interesse an der Tierhaltung, sowie wachsende Bedenken bezüglich des Tierwohls in der Massentierhaltung sogen dafür, dass zunehmend am Einsatz von Informationstechnologie in der Tierhaltungsindustrie geforscht wird. Eine Digitalisierung der Tierhaltung ermöglicht präzisere Entscheidungen und artgerechtere Bedingungen. Beispielsweise kann auf Basis des Tiergewichts die Futtermenge datenbasiert angepasst werden. Eine solche datenbasierte Tierhaltung wird als Präzisionstierhaltung bezeichnet. In diesem Zusammenhang wächst das Interesse am Einsatz von maschinellem Lernen in der Tierhaltung. Mit Lernalgorithmen können Daten maschinell interpretiert werden. Dies ermöglicht eine Automatisierung des Entscheidungsprozesses\cite{Garcia.2020}.\par

Stark im Fokus der Tierforschung liegt dabei die Klassifikation von Tierverhalten. Bestrebungen zur Verhaltensanalyse mittels maschinellem Lernen existieren für die wichtigsten Nutztierarten. In \cite{VazquezDiosdado.2019} werden Lernalgorithmen eingesetzt, um das Verhalten von Schafen zu klassifizieren. In \cite{VazquezDiosdado.2015} wird eine Verhaltensklassifikation von Milchkühen vorgestellt. Auch die Anwendung bei Schweinen wird untersucht \cite{Tran.2023}. Das Tracking von Tieren spielt dabei oftmals eine wichtige Rolle. Tracking ist die Aufzeichnung von Bewegungen individueller Objekte. In Bezug auf die Tierhaltung wird damit die Betrachtung des Individualverhaltens möglich. Während in der Vergangenheit oft Sensoren am Körper die Bewegungen aufzeichneten, wird inzwischen verbreitet am Einsatz von Techniken des maschinellen Sehens geforscht \cite{Cowton.2019, Li.2020, Chen.2023}.\par

Viele Trackingansätze, die für größere Tiere erfolgreich anwendbar sind, sind unpraktikabel für Geflügel. Dennoch wird auch in diesem Bereich geforscht \cite{Chen.2023, Gonzalez.2020, Nasirahmadi.2020}. Herausfordernd in Mastställen für Geflügel sind die hohe Tierdichte, die hohe Ähnlichkeit der Tiere und das schnelle Wachstum, wodurch die Differenzierung und die Zuordnung von Tieren erheblich erschwert wird. Um eine effektive Verhaltensklassifikation mittels maschinellem Sehen zu realisieren, werden Methoden benötigt, welche individuelle Bewegungen sowie Herdenbewegungen korrekt erfassen können. Eine korrekte und robuste Erfassung dieser Bewegungen ermöglicht eine maschinelle Interpretation des Verhaltens. Da aufgrund der komplexen Gegebenheiten solche Erfassungssystem wenig verbreitet sind, ist die maschinelle Interpretation von Geflügelverhalten ein weitgehend unerforschtes Gebiet. \par

Neben diesen technischen Herausforderungen und Hindernissen sind einige Anforderungen zu erfüllen, damit ein System zur Verhaltensinterpretation in der Geflügelindustrie anwendbar ist. Es muss sich für die Anwendung in einem industriellen Maststall eignen. In der Forschung von \cite{Fang.2020, Ju.2021} werden die Systeme zur Bewegungserfassung in Testumgebungen, mit einer geringen Tieranzahl untersucht. Ebenfalls muss es \glsdisp{Echtzeitfähigkeit}{echtzeitfähig} sein, um Reaktionen auf das Verhalten zu ermöglichen. In \cite{Chen.2023, Nasirahmadi.2020} wird die Relevanz dieser Anforderung betont, jedoch sind die resultierenden Systeme nicht echtzeitfähig. \par

In dieser Arbeit wird ein echtzeitfähiges System für die maschinelle Verhaltensklassifikation von Mastputen entwickelt. Es ist auf die Anwendung in einem industriellen Mastputenstall konzeptioniert. Die Entwicklung findet im Rahmen des Forschungsprojekts \textit{\acrshort{OptiLiMa}} (\textit{Optimierung des Lichtmanagements für die Haltung von Mastputen}) statt. Das Projekt wird gefördert von der \textit{landwirtschaftlichen Rentenbank}. Ziel von \textit{\acrshort{OptiLiMa}} ist die Untersuchung des Einflusses der Stallbeleuchtung auf Mastputen. Eine Verhaltensklassifikation kann helfen, den Einfluss zu beurteilen. Kann das Verhalten erfasst werden, so ist ein Vergleich von Verhaltensänderungen und Beleuchtungsänderungen möglich. Relevanz hat ein solches System jedoch generell für die Untersuchung externer Einflüsse auf das Verhalten.\par

Potenzial besteht jedoch nicht nur für die Forschung. Auch auf wirtschaftliche Aspekte sind positive Auswirkungen zu erzielen. In Bezug auf Mastputen erfolgt die Kontrolle und Überwachung der Puten manuell durch Personal. Ein Mensch muss den Stall betreten und einen Kontrollgang vollziehen. Die Puten reagieren auf die Anwesenheit eines Menschen mit hochgradig auffälligem Gruppenverhalten. Hohe Dynamik und ein Gedränge der Puten in Richtung Mensch sind zu beobachten. Diese Abkehr vom Normalverhalten ist eine Stresssituation für die Puten. Durch eine automatische Überwachung des Stallgeschehens können Kontrollgänge reduziert werden, das spart zum einen Arbeitskraft ein und zum anderen entstehen weniger Stresssituationen. Das Tierwohl lässt sich steigern und Ausschuss wird verringert \cite{Chen.2023, Garcia.2020}. Auch Kämpfe unter den Puten sind problematisch für das Tierwohl. Die Puten verletzen sich gegenseitig und die Folgen können tödlich sein. Durch eine Echtzeitverhaltensklassifikation können Kämpfe erkannt werden. Das ermöglicht eine Einflussnahme, wodurch Verletzungen vermieden werden können. Auch eine Erkennung von Krankheiten in der Herde ist denkbar.\par

Die Entwicklung eines Systems zur automatischen Klassifikation von Verhaltensweisen von Mastputen bietet für die Forschung, die Wirtschaft und den Tierschutz hohes Potenzial. Dabei ist ein System zur Verhaltensklassifikation eine Schlüsseltechnologie, um weiterreichende Anwendungen aufzubauen, bis hin zu einem voll automatischen Stallmanagment. Aus diesem Grund wird das System als Modul konzeptioniert. Ein Modul ist eine Teilkomponente eines Systems und ist in sich selbst ebenfalls ein \gls{System}. Es ist eine Subkomponente, welche sich leicht integrieren und austauschen lässt. Die Entwicklung wird somit darauf ausgerichtet, eine einfache Einbindung in weiterreichende Anwendungen, wie z.B. Stallmanagmentsysteme zu ermöglichen. 