\section{Hintergrund und Relevanz der Thematik} \label{sec:Hintergrund}
Produkte tierischem Ursprungs, wie Milch, Eier und Fleisch sind wichtiger Bestandteil der menschlichen Ernährung. Der weltweite Fleischkonsum ist am Steigen. Gerade die Produktion von Geflügel hat in den letzten zwei Jahrzehnten ein enormes Wachstum erlebt \cite{StatistischesBundesamt.20200212}. Durch dieses wirtschaftliche Interesse und die zunehmende Digitalisierung findet zunehmend Informationstechnologie Einzug in die Tierhaltungsindustrie. Dies wird als Präzisionstierhaltung bezeichnet. In der Vergangenheit basierten die getroffenen Entscheidungen in der Tierhaltung vor allem auf Erfahrungswerten des Tierhalters. Die Präzisionstierhaltung zielt auf datenbasierte Entscheidungen ab. In diesem Zusammenhang wächst das Interesse am Einsatz von maschinellem Lernen in der Tierhaltung. \cite{Garcia.2020}.\par

Stark im Fokus der Forschung liegt dabei die Klassifikation von Tierverhalten. Bestrebungen zur Verhaltensanalyse mittels maschinellem Lernen existieren für die wichtigsten Nutztierarten. In \cite{VazquezDiosdado.2019} werden Lernalgorithmen eingesetzt, um das Verhalten von Schafen zu klassifizieren. In \cite{VazquezDiosdado.2015} wird eine Verhaltensklassifikation von Milchkühen vorgestellt. Auch die Anwendung bei Schweinen wird untersucht \cite{Tran.2023}. Das Tracking von Tieren spielt dabei oftmals eine wichtige Rolle. Während in der Vergangenheit oft Sensoren am Körper die Bewegungen aufzeichneten, finden inzwischen Techniken des maschinellen Sehens verbreitet Einsatz.\par

Viele Ansätze, die für größere Tiere erfolgreich anwendbar sind, sind unpraktikabel für Geflügel. Dennoch wird auch in diesem Bereich geforscht \cite{Chen.2023, Gonzalez.2020, Nasirahmadi.2020}. Herausfordernd in Geflügelmastställen sind die hohe Tierdichte, die hohe Ähnlichkeit und das schnelle Wachstum. Das erschwert die Differenzierung und Zuordnung von Tieren erheblich. Um eine effektive Verhaltensklassifikation mittels maschinellem Sehen zu realisieren, werden Methoden benötigt, welche individuelle Bewegungen sowie Herdenbewegungen korrekt erfassen können. Eine korrekte und robuste Erfassung dieser Bewegungen ermöglicht eine maschinelle Interpretation des Verhaltens. Da aufgrund der komplexen Gegebenheiten eine solche Erfassung nicht existiert, ist die maschinelle Interpretation von Geflügelverhalten ein weitgehend unerforschtes Gebiet. \par

Neben diesen technischen Hindernissen muss ein System zur Verhaltensinterpretation einige Anforderungen erfüllen, um Relevanz für die Geflügelindustrie zu haben. Es muss sich für die Anwendung in einem industriellen Maststall eignen. In der existierenden Forschung werden die Systeme zur Bewegungserfassung oftmals in Testumgebungen, mit einer geringen Tieranzahl untersucht \cite{Fang.2020, Ju.2021}. Ebenfalls muss es Echtzeitfähig sein, um Reaktionen auf das Verhalten zu ermöglichen. Auch wenn in der Forschung die Relevanz dieser Anforderung oft betont wird, so sind die resultierenden Systeme nicht echtzeitfähig \cite{Chen.2023, Nasirahmadi.2020}. \par

In dieser Arbeit wird ein echtzeitfähiges System für die maschinelle Verhaltensklassifikation von Mastputen entwickelt. Es ist auf die Anwendung in einem industriellen Mastputenstall konzeptioniert. Die Entwicklung findet im Rahmen des Forschungsprojekts \textit{\acrshort{OptiLiMa}} (\textit{Optimierung des Lichtmanagements für die Haltung von Mastputen}) statt. Das Projekt wird gefördert von der \textit{landwirtschaftlichen Rentenbank}. Ziel von \textit{\acrshort{OptiLiMa}} ist die Untersuchung des Einflusses der Stallbeleuchtung auf Mastputen. Eine Verhaltensklassifikation kann helfen, den Einfluss zu beurteilen. Relevanz hat ein solches System jedoch generell für die Untersuchung externer Einflüsse auf das Verhalten.\par

Potenzial besteht jedoch nicht nur für die Forschung. Auch auf wirtschaftliche Aspekte sind positive Auswirkungen zu erzielen. Die Kontrolle und Überwachung des Stalls und der Mastputen erfolgt in den Betrieben manuell durch Personal. Ein Mensch muss den Stall betreten und einen Kontrollgang vollziehen. Die Tiere reagieren auf die Anwesenheit eines Menschen mit hochgradig auffälligen Gruppenverhalten. Hohe Dynamik und ein Gedränge der Tiere in Richtung Mensch sind zu beobachten. Diese Abkehr vom Normalverhalten ist eine Stresssituation für die Puten. Durch eine automatische Überwachung des Stallgeschehens können Kontrollgänge reduziert werden, das spart zum einen Arbeitskraft ein und zum anderen entstehen weniger Stresssituationen. Das Tierwohl lässt sich steigern und Ausschuss wird verringert \cite{Chen.2023, Garcia.2020}. Auch Kämpfe unter den Puten sind problematisch für das Tierwohl. Die Tiere verletzen sich gegenseitig und die Folgen können tödlich sein. Durch eine Echtzeitverhaltensklassifikation können Kämpfe erkannt werden. Das ermöglicht eine Einflussnahme, wodurch Verletzungen vermieden werden können. Auch eine Erkennung von Krankheiten in der Herde ist denkbar.\par

Es ist zu sehen, dass die Entwicklung eines Systems zur automatischen Klassifikation von Verhaltensweisen von Mastputen Potenzial bietet für die Forschung, die Wirtschaft und den Tierschutz. Dabei ist das System eine Schlüsseltechnologie, um weiterreichende Anwendungen aufzubauen, bis hin zu automatisierten Stallmanagmentsystemen. Aus diesem Grund wird das System als Modul konzeptioniert. Ein Modul ist eine Teilkomponente eines Systems und ist in sich selbst ebenfalls ein \gls{System}. Es ist eine Subkomponente, welche sich leicht integrieren und austauschen lässt. Die Entwicklung wird somit darauf ausgerichtet, eine einfache Einbindung in weiterreichende Systeme zu ermöglichen. 

