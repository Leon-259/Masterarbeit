\clearpage
\section*{Feature-Extraktion aus den Trajektoiren}

\begin{pythoncodeAnhang}{Feature-Extraktino aus den Trajektorien mit \textit{Numba} just-in-time Kompilierung.}
#Initialisierung der Just-in-Time kompilierung
@jit(nopython=True, parallel=True, cache=True)

#Funktion zur Feature-Extraktion
#Jeder Eintrag der Arrays bezieht sich auf eine Detektion

#trajIDs:        Array der IDs der Trajektorien
#frames:         Array mit den Nummern der Frames
#unixtimestamps: Array mit den Unixzeitstempeln zu den Frames
#x_vals:         Array mit den X werten zu den Detektionen
#y_vals:         Array mit den X werten zu den Detektionen
def __get_features(self, trajIDs, frames, unixtimestamps, 
                x_vals, y_vals):
    
    #Höchste Trajektorien ID. Wird für die Iteration benötigt
    trajIDmax = np.nanmax(trajIDs)

    #Initalisierungen der Arrays für die Feature-Werte
    #Für jede Trajektorie wird ein Eintrag initialisiert
    #Bei der Kompilierung muss vorher feststehen wie viel
    #Wieviel Speicherplatz benötigt wird.
    geschwind = np.full((trajIDmax, 1), np.nan, np.float32)
    beschleun = np.full((trajIDmax, 1), np.nan, np.float32)
    stepCount = np.full((trajIDmax, 1), np.nan, np.float32)
    distanzen = np.full((trajIDmax, 1), np.nan, np.float32)
    strecken = np.full((trajIDmax, 1), np.nan, np.float32)
    kurvigkeiten = np.full((trajIDmax, 1), np.nan, np.float32)
    x_geschw = np.full((trajIDmax, 1), np.nan, np.float32)
    y_geschw = np.full((trajIDmax, 1), np.nan, np.float32)

    #Starten der Iterationen durch die Trajektorien
    #prange: Numba-Funktion, parallelisiert die For-Schleife 
    for id in prange(trajIDmax):

        #ID-Indizes aus trajIDs zu der betrachteten Trajektorie
        idx = np.atleast_1d(np.where(trajIDs == id)[0])

        #Frame Nummer zu den Einträgen der Trajektorie
        framesTraj = frames[idx]    
        #Unixzeiten zu den Einträgen der Trajektorie
        unixtimestampsTraj = unixtimestamps[idx] 
        #X-Werte & Y-Werte zu den Einträgen der Trajektorie
        x_valsTraj = x_vals[idx]
        y_valsTraj = y_vals[idx]

        #Differenzen zwischen den X-Werten (Y-Werten)
        #Schrittweiten bzw. Partielle Ableitung nach x (y)
        x_diffsTraj = np.diff(x_valsTraj)
        y_diffsTraj = np.diff(y_valsTraj) 

        #Zeitschritte in Sekunden
        timeDiffsTraj = np.diff(np.divide(unixtimestampsTraj,\
                                1000))        

        #Zurückgelegte Distanzen in den einzelnen Schritten
        streckeStepsTraj = np.sqrt((x_diffsTraj**2) + \
                                    (y_diffsTraj**2))

        #Geschwindigkeiten & Beschleunigung in jedem Schritt
        geschwindigkeitenTraj = np.divide(streckeStepsTraj, \ 
                                        timeDiffsTraj)
        beschleunigungenTraj = np.divide(geschwindigkeitenTraj, \
                                        timeDiffsTraj)
        
        #Mittlere Geschwindigkeit der Trajektorie
        geschwind[id] = np.nanmean(geschwindigkeitenTraj)
        
        #Mittlere Beschleunigung der Trajektorie#
        beschleun[id] = np.nanmean(beschleunigungenTraj)
        
        #Schrittanzahl der Trajektorie
        stepCount[id] = np.nan

        #Distanz: Euklidischer Abstand von Startpunkt zu Endpunkt
        distanzen[id] = np.sqrt((np.nansum(x_diffsTraj)**2) + \
                                (np.nansum(y_diffsTraj)**2))  
        
        #Strecke: Bewegungslinie der Trajektorie
        strecken[id] = np.nansum(streckeStepsTraj)
        
        #Kurvigkeit: Strecke/Distanz
        #Ist die Strecke nur auf Schwankungen der Lokalisation 
        #Zurückzuführen, wird die Kurvigkeit auf 1 gesetzt
        if strecken[id] < 2*np.max(framesTraj):
            kurvigkeiten[id] = 1
        else:
            kurvigkeiten[id] = strecken[id]/distanzen[id]

        #Geschwindigkeit in x (y) Richtung
        x_geschw[id] = \
            np.nansum(x_diffsTraj)/np.nansum(timeDiffsTraj)
        y_geschw[id] = \
            np.nansum(y_diffsTraj)/np.nansum(timeDiffsTraj)

    #Feature-Extraktion. Komprimieren der Zeitreihen der 
    #Trajektorien zu den Features

    #Geschwindigkeitsfeatures
    meanGeschwindigkeiten = np.nanmean(geschwind)
    stdGeschwindigkeiten = np.nanstd(geschwind)
    maxGeschwindigkeiten = np.nanmax(geschwind)
    minGeschwindigkeiten = np.nanmin(geschwind)

    #Drift-Geschwindigkeit-Feature
    x_drift = np.nansum(x_geschw)
    y_drift = np.nansum(y_geschw)
    drift_geschw = np.sqrt((x_drift**2) + (y_drift**2))

    #Beschleunigungsfeatures
    meanBeschleunigungen = np.nanmean(beschleun)
    stdBeschleunigungen = np.nanstd(beschleun)
    maxBeschleunigungen = np.nanmax(beschleun)
    minBeschleunigungen = np.nanmin(beschleun)

    #Schrittanzahl-Features
    meanStepCount = np.nanmean(stepCount)
    stdStepCount = np.nanstd(stepCount)
    minStepCount = np.nanmin(stepCount)

    #Distanz-Features
    meanDistanzen = np.nanmean(distanzen)
    stdDistanzen = np.nanstd(distanzen)
    maxDistanzen = np.nanmax(distanzen)

    #Strecke-Features
    meanStrecken = np.nanmean(strecken)
    stdStrecken = np.nanstd(strecken)
    maxStrecken = np.nanmax(strecken)
    minStrecken = np.nanmin(strecken)
    totalStrecke = np.nansum(strecken)

    #Kurvigkeit-Features
    meanKurvigkeiten = np.nanmean(kurvigkeiten)
    stdKurvigkeiten = np.nanstd(kurvigkeiten)
    maxKurvigkeiten = np.nanmax(kurvigkeiten)

    #Rückgabe der Feature-Werte
    return  meanGeschwindigkeiten, \
            stdGeschwindigkeiten, \
            maxGeschwindigkeiten, \
            minGeschwindigkeiten, \
            meanBeschleunigungen, \
            stdBeschleunigungen, \
            maxBeschleunigungen, \
            minBeschleunigungen, \
            meanStepCount, \
            stdStepCount, \
            minStepCount, \
            meanDistanzen, \
            stdDistanzen, \
            maxDistanzen, \
            meanStrecken, \
            stdStrecken, \
            maxStrecken, \
            minStrecken, \
            totalStrecke, \
            meanKurvigkeiten, \
            stdKurvigkeiten, \
            maxKurvigkeiten, \
            drift_geschw
\end{pythoncodeAnhang}