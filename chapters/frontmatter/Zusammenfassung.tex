\section*{Zusammenfassung}

Durch die steigende Produktion von Geflügel wächst auch das Interesse an der Digitalisierung der Tierhaltung. Stark von Interesse ist die maschinelle Klassifikation von Tierverhalten. In dieser Arbeit wird untersucht, wie mit Methoden des maschinellen Lernens ein Modul entwickelt werden kann, welches unerwünschtes Verhalten bei Mastputen automatisch detektiert. Ein solches System verspricht wirtschaftliche Einsparungen durch reduzierten Ausschuss und Arbeitsaufwand, wie auch eine Steigerung des Tierwohls. Unerwünschtes Verhalten sind Kämpfe zwischen den Tieren und hohe Gruppendynamiken, wie sie bei Kontrollgängen des Tierhalters vorkommen. Diese Verhaltensweisen werden mit einer Support Vector Machine (SVM) klassifiziert. Um dieses Modell des maschinellen Lernens anzuwenden, wird ein Machine Learning Workflow befolgt. Dieser beginnt mit dem Sammel und Vorverarbeiten von Trainingsdaten. Mit einem Tool findet eine Verifikation der Daten statt, um die Qualität zu sichern. Anschließend werden Features aus den Daten entwickelt. Mit den Features wird die SVM eingestellt, trainiert und getestet. Es wird in das Modul integriert. Mit einer Simulation der Anwendung im Mastputenstall wird das Modul evaluiert. Obwohl das Modell im Test kein Overfitting aufweist, ist das Modul in der Simulation nicht in der Lage Kämpfe von Normalverhalten zu unterscheiden. Das deutet auf eine zu geringe Datenmenge hin und zu unspezifische Features. Kontrollgänge werden hingegen erkannt. Dies validiert das Vorgehen entlang des Machine Learning Workflows. Das Modul ist Echtzeitfähig und lässt sich durch das modulare Konzept einfach in weiterführende Anwendungen, wie z.B. eine Stallregelung integrieren. 

\todo{Relevanz hinzufügen}
\todo{Optilima -> Frage aus Übberprojekt ableiten}
\todo{Weg System Modul}
\todo{Stallmanagment nicht Regelung}