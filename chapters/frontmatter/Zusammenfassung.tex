\section*{Zusammenfassung}
\vspace*{-5mm}
Durch die steigende Produktion von Geflügel wächst auch das Interesse an der Digitalisierung der Tierhaltung. Stark von Interesse ist die maschinelle Klassifikation von Tierverhalten. Grundlage dafür ist eine Erfassung von individuellem Verhalten und von Gruppenverhalten einer Herde. Das ist mit Multi-Objekt Tracking (MOT) möglich. Die Anwendung von MOT bei Geflügel ist durch eine hohe Tierdichte und große Ähnlichkeiten besonders herausfordernd. Aus diesem Grund existieren keine Systeme, welche in der Lage sind, Verhalten in einem industriellen Mastputenstall automatisch zu klassifizieren. In dieser Arbeit wird untersucht, wie mit Methoden des maschinellen Lernens ein System entwickelt werden kann, welches unerwünschtes Verhalten bei Mastputen automatisch detektiert. Sie ist Teil des Forschungsprojekts \textit{Optimierung des Lichtmanagements für die Haltung von Mastputen}. Das System soll die Untersuchung des Einflusses der Stallbeleuchtung ermöglichen. Es wird als Modul konzeptioniert, um eine einfache Einbindung in ein automatisches Stallmanagment zu ermöglichen. Unerwünschtes Verhalten sind Kämpfe zwischen den Tieren und hohe Gruppendynamiken, wie sie bei Kontrollgängen vorkommen. Die Verhaltensweisen klassifiziert eine Support Vector Machine (SVM). Um dieses Modell des maschinellen Lernens anzuwenden, wird ein Machine Learning Workflow befolgt. Dieser beginnt mit dem Sammel und Vorverarbeiten von Trainingsdaten. Mit einem Tool findet eine Verifikation der Daten statt, um die Qualität zu sichern. Anschließend werden Features aus den Daten entwickelt. Mit den Features wird die SVM eingestellt, trainiert und getestet. Es wird in das Modul integriert. Mit einer Simulation der Anwendung im Mastputenstall wird das Modul evaluiert. Obwohl das Modell im Test kein Overfitting aufweist, ist das Modul in der Simulation nicht in der Lage, Kämpfe von Normalverhalten zu unterscheiden. Das deutet auf eine zu geringe Datenmenge hin und zu unspezifische Features. Kontrollgänge werden hingegen erkannt. Dies validiert das Vorgehen entlang des Machine Learning Workflows. Das Modul ist echtzeitfähig und lässt sich durch das modulare Konzept einfach in weiterführende Anwendungen integrieren. 
