\section*{Abstract}
With the increasing production of poultry, interest in the digitalization of animal farming is also growing. Research focuses on the use of machine learning for the classification of animal behavior. This work investigates how a module can be developed using machine learning methods to detect undesirable behavior in turkeys automatically. Such a system promises economic savings through a reduction of animal deaths and labor, as well as an enhancement of animal welfare. Undesirable behaviors are fights between turkeys and high group dynamics, which occur during inspections of the farmer. These behaviors are classified using a Support Vector Machine (SVM). A Machine Learning Workflow is followed to apply this machine learning model. The Workflow starts with the collection and preprocessing of training data. A tool is used for the verification of the data to ensure quality. From this data, features are developed which are used to  configure, train, and test the SVM. afterward the SVM is integrated into the module. The module is evaluated with a simulation of the application in a turkey barn. Although the model does not show overfitting in the test, the module is unable to distinguish fights from normal behavior in the simulation. This indicates an insufficient amount of data and too unspecific features. Inspections of the farmer are recognized. This validates the approach along the Machine Learning Workflow. The module is capable of real-time operation and can be easily integrated into further applications, such as barn regulation system.