\section*{Abstract}
\vspace*{-5mm}
With the increasing production of poultry, interest in the digitalization of animal farming is also growing. Research focuses on the use of machine learning for the classification of animal behavior. To capture individual and group movements of the flock is the Basis. This can be achieved with multi-object tracking (MOT). The use of MOT in poultry farming is challenging due to high animal density and high similarity. For this reason, no systems exist which are able to automatically classify behavior in industrial poultry farming setting. This work investigates how a module can be developed using machine learning methods to detect undesirable behavior in turkeys automatically. It is part of the research project \textit{Optimization of Light Management in poultry farming}. The system enables the investigation of the influence of barn lighting. It is conceptualized as a module to allow easy integration into an automatic, barn management system. Undesirable behaviors are fights between turkeys and high group dynamics, which occur during inspection walks. These behaviors are classified using a Support Vector Machine (SVM). A Machine Learning Workflow is followed to apply this machine learning model. The Workflow starts with the collection and preprocessing of training data. A tool is used for the verification of the data to ensure quality. From this data, features are developed which are used to  configure, train, and test the SVM. afterward, the SVM is integrated into the module. The module is evaluated with a simulation of the application in a turkey barn. Although the model does not show overfitting in the test, the module is unable to distinguish fights from normal behavior in the simulation. This indicates an insufficient amount of data and too unspecific features. Inspections of the farmer are recognized. This validates the approach along the Machine Learning Workflow. The module is capable of real-time operation and can be easily integrated into further applications.


