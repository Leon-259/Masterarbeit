\section{Rückblick auf das Forschungsvorhaben und erreichte Ziel}

Die für die Umsetzung des Forschungsvorhabens gewählten Methoden wurden alle mit sorgfältiger Literaturarbeit entwickelt. Das Vorgehen orientierte sich stets an den in der Literatur verwendeten Prozessen und Methoden. Der Machine Learning Workflow diente als Leitpfaden für das Vorgehen. Innerhalb der Prozessschritte des Machine Learning Workflows sind teilweise eigene Methoden entwickelt worden. Ein Beispiel ist die Vorauswahl der Modelle. In der betrachteten Literatur zum maschinellen Lernen wird keine explizite Methode zur Auswahleingrenzung genannt. Die Anwendung von klassischen Methoden der Entscheidungsfindung schien jedoch zielführend. Mit der Entwicklung des Wrapper-Scores wurde ein Versuch unternommen, eine Feature-Auswahl multivariat und modellunabhänig zu beurteilen. Das Vorgehen ist, aufgrund des hohen Aufwands der Erhebung der Daten, aus denen sich der Wrapper-Score berechnet, in vielen Fällen unpraktikabel. Im Falle der Arbeit war hier die geringe Datenmenge ein Vorteil. Für die  multivariate und modellunabhänige Beurteilung von Features existieren in der Literatur keine verbreiteten Ansätze. \par

Von großem Vorteil für die Arbeit war Fachwissen. Im Rahmen des \acrshort{OptiLiMa} Projekts wurde sich bereits intensiv mit dem Verhalten der Tiere befasst. Dadurch konnte auf Wissen, Anwendungen und Daten zurückgegriffen werden. Das sorgte für eine große Zeitersparnis in vielerlei Hinsicht. Auch ermöglichte das Fachwissen eine sehr spezifische Feature-Extraktion und Konstruktion. \par

In Bezug auf die Datenmenge zum Modellaufbau war Zeit ein entscheidender Faktor. Von Beginn an war bekannt, dass die Datenmenge sehr klein ist für Lern-Algorithmen. Durch Vorerfahrung war jedoch die Zeitlaufwendigkeit der manuellen Suche nach Kämpfen abschätzbar. Aus diesem Grund wurden keine weiteren Daten ermittelt. Stattdessen fiel der Fokus auf die Verifizierung der vorhandenen Daten und damit auf die Sicherstellung der Datenqualität. Der verfolgte Ansatz war somit Qualität über Quantität. Die Erkennung der Kontrollgänge gibt diesem Ansatz recht, während die Erkennung der Kämpfe zeigt, dass maschinelles Lernen nicht ganz ohne Quantität funktioniert. \par

Ein weiterer Faktor, welcher die Zeitplanung negativ beeinflusste, war die Bewertung des MOT Systems. Dies war ursprünglich nicht als Teil dieser Arbeit vorgesehen. Die Evaluation sollte im Rahmen einer anderen Arbeit stattfinden. Diese Arbeit brachte jedoch keine Ergebnisse hervor. Da eine Einschätzung der Qualität der Trajektorien große Relevanz für das Feature-Engineering hatte und generell für das \acrshort{OptiLiMa} Projekt, wurde sich dafür entschieden, die Zeit aufzubringen, um die Evaluation durchzuführen. Die meiste Zeit floss dabei in die Recherche zu den MOT Metriken. Vorteilhaft war, dass die Ground Truths durch die andere Arbeit vorhanden waren, sonst wäre eine Evaluation zeitlich nicht möglich gewesen.\par

Die intensive Auseinandersetzung mit der Literatur, um passende Methoden und Ansätze für mein Vorgehen zu finden, hat meine Fähigkeiten in der Arbeit mit Literatur sehr gesteigert. Gerade die Auseinandersetzung mit jungen Forschungsfeldern wie MOT war zu Beginn herausfordernd, da so gut wie keine Lehrbücher existieren. Dadurch verbesserten sich meine Fähigkeiten, relevante wissenschaftliche Veröffentlichungen zu finden und zu verstehen. Das hatte wiederum positiven Einfluss auf meine Ausarbeitung, da ich mehr Gespür für einen wissenschaftlichen Schreibstil und für wissenschaftliches Vorgehen entwickelte. Künstliche Intelligenz und maschinelles Lernen entwickeln seit einigen Jahren enormes kommerzielles Potenzial. Sich mit diesen sehr aktuellen Themen auseinanderzusetzen, hat mir viel Freude bereitet. Ebenfalls laufen in dieser Arbeit viele Ideen, Überlegungen und Erkenntnisse aus dem \acrshort{OptiLiMa} Projekt zusammen. Da \acrshort{OptiLiMa} nun beendet ist, freut es mich, mit dieser Arbeit einige der Visionen zu Ende geführt zu haben.\par


\subsection{Evaluierung der gesetzten Ziele}

\textbf{Modularität}\\
Die Modularität ist erfüllt, das zeigt das entwickelte Konzept in \ref{sec:Meth FinalKonzept} und auch in der Umsetzung des Moduls (\ref{sec:umse ModulKonzept})  wird die modulare Integration in eine Datenverarbeitungskette deutlich. \par

\textbf{Echtzeitfähigkeit}\\
Die Echtzeitfähigkeit ist gegeben. Das Modul erfüllt die zeitlichen Anforderungen, wie in \ref{sec:Ergeb Sim} dargestellt.\par

\textbf{Verhaltenserkennung}\\
Die Verhaltenserkennung erfüllt das Ziel nicht zufriedenstellend. Zwar zeigt die Erkennung der Kontrollgänge, dass das Modul Verhalten teilweise richtig erkennen kann, jedoch scheitert es an den Kampfereignissen. Das Modul kann einen Kampf nicht von Normalverhalten unterscheiden. \par

\textbf{Verwendung von maschinellem Lernen}\\
Das Ziel wurde erfolgreich umgesetzt. Die ausführliche Darstellung der Anwendung des Machine Learning Workflows zeigt, wie im Kontext einer Verhaltensklassifizierung von Mastputen maschinelles Lernen verwendet wird. Auch im Konzept des Moduls ist die Anwendung eines Modells des maschinellen Lernens deutlich zu erkennen (\ref{sec:Meth FinalKonzept}). \par

\textbf{Zuverlässigkeit}\\
Das Modul konnte in der Simulation ohne Fehler im Programmablauf zwei volle Tage auswerten. Somit ist die Zuverlässigkeit des Konzepts bestätigt. Langzeittests sind jedoch nicht durchgeführt worden. Somit ist das Ziel mit Vorbehalt als erreicht zu beurteilen. 
\par

\textbf{Automation}\\
Auch dieses Ziel ist erreicht. Das bestätigt die Simulation und das erreichte Ziel der Zuverlässigkeit.\par



\subsection{Beantwortung der Forschungsfrage}
Das Vorgehen in der Arbeit wurde auf die Beantwortung der Forschungsfrage strukturiert. Die definierten Ziele lassen sich direkt aus der Forschungsfrage ableiten. Aus den Zielen ergaben sich die Anforderungen an das Modul und über die Anforderungen ließ sich die Aufgabe definieren und das Konzept erstellen. Das Konzept ist somit aus der Forschungsfrage abgeleitet und dient der Beantwortung. \par

Durch ausführliche Literaturrecherchen wurden Methoden und Verfahren, für die Realisierung des Konzepts gesucht, welche sich in der Praxis bewährt haben. Somit erfolgten die Literaturrecherchen ebenfalls vor dem Hintergrund der Beantwortung der Forschungsfrage. Die Erkenntnisse sind im finalen Konzept zusammengeführt worden. Die Anwendung des Moduls in der Simulation ist der Versuch, die Forschungsfrage zu beantworten. Aus der Auswertung ergeben sich folgende Erkenntnisse.\par

Die Forschungsfrage ließ sich teilweise beantworten. Da mit den angewendeten Methoden des maschinellen Lernens Kontrollgänge erkannt werden können, wird gezeigt, dass unerwünschtes Verhalten detektiert werden kann. Die Erkennung von Kämpfen funktioniert nicht. Somit sind nicht alle unerwünschten Verhaltensweisen erkennbar.\par

Die Erkennung der Kontrollgänge validiert das prinzipielle Vorgehen. Die Unfähigkeit Kämpfe zu erkennen, zeigt jedoch, dass die getroffenen Entscheidungen zu keiner Antwort führen, in Bezug auf die Frage, wie unerwünschtes Verhalten jeglicher Art erkannt werden kann. 