\section{Zusammenfassung der Ergebnisse}
Untersucht wurde, ob sich unter Anwendung von Methoden des maschinellen Lernens ein Modul entwickeln lässt, welches die Detektion von unerwünschtem Verhalten bei Mastputen ermöglicht. Das Vorgehen zur Entwicklung des Moduls zur Verhaltensklassifikation orientierte sich stark am Machine Learning Workflow. Das sorgte für ein strukturiertes Vorgehen mit Methoden, welche sich in der Praxis bewährt haben.\par

Es stand nur eine kleine Menge an Ereignissen zur Verfügung, für den Aufbau eines lernenden Modells. Um die Datenqualität sicherzustellen, wurde ein Konzept für ein Tool entwickelt, mit dem sich die Ereignisse verifizieren lassen und gelabelt werden können. Das Tool ermöglichte eine effiziente und fehlertolerante Verifizierung der Ereignisse. Die Zerteilung der Ereigniszeiträume in Intervalle von 40 Sekunden, ermöglichte ein technisch realisierbares Konzept des Moduls.\par

Die Evaluation des MOT Systems ermöglichte wichtige Einsichten in die Performance der Assoziationsalgorithmen. Der SORT Algorithmus ist besser als der Crocker-Grier Linking Algorithmus. Mit steigender Dynamik sinkt die Performance jedoch stark. Da Features aus den Trajektorien ausgewählt wurden und da die hochdynamischen Kontrollgänge erkannt werden, ist bestätigt, dass die Trajektorien unabhängig von der Verhaltensweise Informationen beinhalten, mit denen Ereignisse unterschieden werden können.\par

Die Eingrenzung der Auswahl der Lern-Algorithmen mittels einer Nutzwertanalyse ermöglichte eine effektive Entscheidungsfindung. Da die Vermeidung von Subjektivität in einer Nutzwertanalyse schwierig ist, wurde nur eine grobe Eingrenzung vorgenommen. Die Eingrenzung fiel auf einfache Modelle, welche überwacht lernen. Um die Zeitreihen der Ereignisdaten für die Modelle verständlich zu machen, werden aus diesen Features extrahiert, welche die Informationen zu einem einzelnen Datenpunkt komprimieren. \par

Für die Feature-Extraktion wurde Fachwissen genutzt, um gezielten Bias zu schaffen, welcher die charakteristischen Merkmale der Verhaltensweisen in den Features hervorheben soll. Die Feature-Auswahl erfolgte mit einer Vorgehensweise, die aus den Wrapper-Methoden abgeleitet wurde. Das Besondere an der Vorgehensweise ist, dass sie eine multivariate und modellabhängige Beurteilung der Features ermöglicht und Redundanzen reduziert. \par

Als bestes Modell wurde eine polynomiale SVM bestimmt, wobei die Performanceunterschiede zwischen den Modellen nur gering waren. Der Vergleich von der Accuracy im Training und der Accuracy im finalen Test zeigte kein Overfitting und kein Underfitting.\par

Die Simulationsergebnisse zeigen, dass das Modul nicht in der Lage ist Kämpfe zu erkennen. Die Gründe liegen vermutlich in der geringen Datenmenge, der Komprimierung der Informationen der Ereignisdaten und Features, die für die Unterscheidbarkeit der Verhaltensweisen unzureichend sind. Kontrollgänge kann das Modul detektieren, auch wenn eine Überempfindlichkeit vorhanden ist. \par

Die Entwicklung des Moduls zeigt, dass eine automatische Erkennung von Verhaltensweise mittels Methoden des maschinellen Lernens prinzipiell möglich ist. Jedoch benötigt es Features, welche die charakteristischen Merkmale der Verhaltensweisen gut erfassen und die Verhaltensweisen unterscheidbar machen. Mit einer hohen Datenqualität können Komplikationen durch eine geringe Datenmenge teilweise kompensiert werden. Dennoch benötigt maschinelles Lernen eine gewisse Grundmenge an Daten, um zuverlässige Modelle zu erstellen. 