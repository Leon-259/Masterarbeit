\section{Ausblick} \label{sec:Ausblick}

Die Arbeit zeigt, dass sich mit der Vorgehensweise eine Echtzeit-Verhaltensklassifikation realisieren lässt. Um ein Modul zu erhalten, welches für den Anwendungsfall einsatzfähig ist, muss jedoch zu einigen Prozessen im Machine Learning Workflow zurückgekehrt werden. \par

Fundamental ist die Datenmenge. Für einen erneuten Versuch sollte die Datenmenge erhöht werden, um die Generalisierbarkeit des Modells zu verbessern und dieses auch besser beurteilen zu können. Dazu werden insgesamt mehr Daten benötigt, jedoch sollte auch die Varianz in den Daten erhöht werden. Das Sammeln der Daten sollten in unterschiedlichen Stallbereichen, Mastdurchläufen und im Idealfall sogar aus unterschiedlichen Ställe stattfinden. \par

Ebenfalls sind neue Features zu erschaffen. Mit den verwendeten Features lassen sich Kämpfe nicht vom Normalverhalten unterscheiden. Dazu können weitere kreative Features aus den bereits vorhandenen Datenquellen generiert werden. Bisher vorhandene Quellen sind das Video, die Detektionen und die Trajektorien, wobei diese alle auf dem Video aufbauen. Es können jedoch auch neue Datenquellen geschaffen werden, aus denen sich Features extrahieren lassen. Ein Beispiel können Audioaufnahmen aus dem Stall sein. \par

Ein weiteres Hindernis für die Erkennung von Kämpfen ist die Komprimierung der Zeitreihendaten. Langfristig kann eine Neuevaluation der Modellauswahl sinnvoll sein, um zu prüfen, ob Deep Learning Modelle mit Sequenzverständnis angewendet werden können. Ein Modell mit Sequenzverständnis hätte mehrere Vorteile. Informationen zu individuellen Trajektorien und lokalen Dynamiken würden nicht in einer Komprimierung verloren gehen. Auch könnte der zeitliche Versatz zwischen Ereignisbeginn und Klassifikationsergebnis reduziert werden. \par

Ein funktionierendes Modul für die Verhaltensklassifikation ermöglicht weitere Forschungsansätze. Dazu zählt beispielsweise die Ursachenforschung. Wenn sich eine Verhaltensweise zuverlässig erkennen lässt, kann untersucht werden, was die Auslöser sind und ob sich das Verhalten regeln lässt. In Bezug auf die Präzisionstierhaltung wären diese Erkenntnisse wegweisend für die Digitalisierung und Automatisierung der Tierhaltungsindustrie.