\section{Bewertung des MOT-Systems} \label{sec:Meth MOT}
Mit einem \gls{MOT} System sollen \gls{Trajektorie}[n] generiert werden, um diese für \gls{Feature}[s] zu nutzen. In \ref{sec:Meth DefAufgabe} wird dargestellt, dass die Merkmale der Verhaltensweisen stark auf der Bewegung der Puten basieren. Über die \gls{Trajektorie}[n] sollen diese Informationen extrahiert werden können. Um das zu gewährleisten, muss das \gls{MOT} System gut genug sein, um die Bewegungsmerkmale jeglicher Verhaltensweisen zu erfassen. Dies ist durch die Bewertung des \gls{MOT} Systems zu überprüfen. \par

Für die Evaluation des \gls{MOT} Systems ist eine \gls{Ground Truth} notwendig. Da die Evaluation Gültigkeit für einen sehr konkreten Anwendungsfall haben soll, ist es sinnvoll, eine anwendungsbezogene \gls{Ground Truth} zu verwenden anstatt Benchmarking-Datensätze. \gls{Ground Truth}[s] für zwei Ereignisse wurden bereits vor dieser Arbeit erstellt. Wie in \ref{sec:MOT GT} dargestellt, ist die Erstellung einer qualitativ hochwertigen \gls{Ground Truth} herausfordernd. Um die Aussagekraft der Evaluation beurteilen zu können, wird die Qualität der \gls{Ground Truth} diskutiert. \par

Für das \glsdisp{Assoziation}{Assoziationsmodul} ist ein Algorithmus auszuwählen. Dazu ist zwischen dem \acrshort{SORT} Algorithmus (\autoref{sec:MOT SORT}) und einer modifizierten Variante des Crocker-Grier Linking Algorithmus (\autoref{sec:MOT CrockGrier}) abzuwägen. Die Modifikationen werden in \autoref{sec:Meth CrockGrieMods} erläutert. Für das Gesamtsystem ist zu bewerten, welcher der Algorithmen die besten \gls{Feature}[s] gewährleisten kann. \par

Während der Versuche werden die Laufzeiten der Algorithmen gemessen, da diese  in Bezug auf die Echtzeitfähigkeit des Moduls relevant sind. 


\subsection{Diskussion der Ground Truth Qualität} \label{sec:Meth GT Quali}
Noch vor dieser Arbeit sind zu zwei Ereignissen \gls{Ground Truth}[s] erstellt worden:  ein Kampfereignis und ein Kontrollgangereignis. Die Ereignisse sind jeweils eine Minute lang. Das Kampfereignis besteht aus 91 \gls{Frame}[s] und das Kontrollgangereignis umfasst 98 \gls{Frame}[s]. Dadurch ergibt sich eine durchschnittliche \gls{Frame}[rate] von 1,6 \gls{Frame}[s] pro Sekunde. Die Ereignisse wurden von unterschiedlichen Personen bearbeitet.  \par

Die Herausforderungen, die die Erstellung einer qualitativ hochwertigen \gls{Ground Truth} mit sich bringt, waren zum Zeitpunkt der Erstellung innerhalb des \textit{\acrshort{OptiLiMa}} Projekts unbekannt. Aus diesem Grund wurden nur wenige Regeln festgelegt. Es wurde definiert, dass jedes Tier mit einer rechteckigen \gls{Bounding Box} zu \glsdisp{Detektion}{detektieren} ist. Die \gls{Bounding Box} soll das Tier möglichst eng und vollständig umschließen. Tiere sollen möglichst auch nach Verdeckungen korrekt \glsdisp{Assoziation}{assoziiert} werden. Wie mit Mehrdeutigkeiten und anderen komplizieren, Situationen umzugehen ist, wurde nicht geregelt. \par 

Bei der Überprüfung der \gls{Ground Truth} fiel auf, die \gls{Detektion} funktionierte gut. Auch die \gls{Lokalisation} war zufriedenstellend, auch wenn Schwankungen zu erwarten sind. Schwächen sind vor allem bei Mehrdeutigkeiten vorhanden. Die \gls{Detektion} bei teilweisen Verdeckungen war inkonstant. Auch haben die unterschiedlichen Bearbeiter mehrdeutige Situationen unterschiedlich gelöst. Dadurch ist anzunehmen, dass Unsicherheiten in der Bewertung des \gls{Detektion}[smoduls] entstehen. Ein Vergleich der Detektionen des Detektionsmoduls und der Ground Truth Detektionen zeigte, dass die \gls{Ground Truth} übermäßig optimistische Erwartungen an das \gls{Detektion}[smodul] stellt. Bei teilweisen Verdeckungen wurde eine Pute in der Ground Truth länger \glsdisp{Detektion}{detektiert}, als vom \gls{Detektion}[smodul]. Somit ist mit einer vermehrten Anzahl von falsch negativen \gls{Detektion}[en] (\autoref{sec:MOT Fehlertypen}) zu rechnen. \par

Die niedrige \gls{Frame}[rate] sowie die hohe Tierdichte im Stall sorgten dafür, dass sich Tiere teilweise nur schwer auseinanderhalten ließen. Dadurch sind Unsicherheiten in der Evaluation des Assoziationsmoduls wahrscheinlich. 

Es ist empfehlenswert, dass das \gls{Detektion}[smodul] mit Daten trainiert wird, die aus der gleichen Quelle stammen wie die \gls{Ground Truth}. Das stellt eine faire Bewertung sicher, da keine Diskrepanz vorhanden ist zwischen dem, was das \gls{Detektion}[smodul] lernt und dem, was die Evaluation von ihm fordert. Dies war hier nicht möglich, da die \gls{Detektion} bereits abgeschlossen war, als die \gls{Ground Truth} erstellt wurde. Dadurch ist mit Unsicherheiten der Bewertung des \gls{Detektion}[smoduls] und des \gls{Lokalisation}[smoduls] zu rechnen.\par

Um trotz der Fehler in der Ground Truth stabile Ergebnisse zu erhalten, wird empfohlen, mehrere \gls{Ground Truth}[s] zu verwenden und die Ergebnisse zu mitteln. Für die Bewertung bezogen auf den Anwendungsfall stehen jedoch nur zwei \gls{Ground Truth} Datensätze zur Verfügung. Das kann die Stabilität der Wertungen negativ beeinflussen.


\subsection{Modifikation des Crocker-Grier Linking Algorithmus} \label{sec:Meth CrockGrieMods}
Der grundlegende Aufbau des Crocker-Grier Linking Algorithmus ist in \autoref{sec:MOT CrockGrier} erläutert. Die Modifikationen erfolgten bereits vor dieser Arbeit. Sie werden hier angeführt, um die Nachvollziehbarkeit der Arbeitsweise zu ermöglichen. \par

Bei der Anwendung der ursprünglichen Form des Crocker-Grier Linking Algorithmus auf Ereignisse im Putenmaststall kam es zu Problemen. Die maximale Distanz \(l\), die ein Objekt zwischen zwei \gls{Frame}[s] zurücklegen darf, um korrekt \glsdisp{Assoziation}{assoziiert} werden zu können, wird von den Tieren oftmals nicht eingehalten. Gerade bei dynamischen Ereignissen legen die Puten in kürzerer Zeit weitere Strecken zurück. Dies ist auch der niedrigen \gls{Frame}[rate] und hohen Tierdichte geschuldet. Durch die Verletzung von \(l\) entstehen \gls{Assoziation}[sfehler] (\autoref{sec:MOT Fehlertypen}). Durch die Vergrößerung von \(l\) entstehen größere Netzwerke. Das führte, in der Anwendung teilweise zu unpraktikablen Rechenzeit.\par

Um diesen Problemen zu begegnen, wurde eine iterative \gls{Assoziation} umgesetzt. Diese durchläuft mehrere Stufen, in denen die \gls{Assoziation} wiederholt wird. Bei jeder Iteration wird \(l\) vergrößert. Korrekte \gls{Assoziation}[en] werden in den folgenden Stufen nicht mehr betrachtet. Insgesamt werden drei Iterationen durchlaufen. Die Abbildung \ref{fig:funkCrockGrierMod} zeigt die Auswirkungen dieses Ansatzes. In der ersten Stufe werden vor allem stationäre Puten \glsdisp{Assoziation}{assoziiert}. Diese werden in den folgenden Stufen nicht mehr betrachtet. Dadurch wird der Abstand zwischen den zu \glsdisp{Assoziation}{assoziierenden} Objekten größer und korrekte \gls{Assoziation}[en] sind auch mit einem größeren \(l\) zu erwarten. Auch die Ausmaße der entstehenden Netzwerke sind besser handhabbar. 

\begin{figure}[htb]
    \centering
    \includegraphics[width=0.9\textwidth]{img/Grafiken/CrockGrierMod Funktionsprinzip.png}
    \caption[Ansatz der Modifikationen des Crocker-Grier Linking Algorithmus.]{Ansatz der Modifikationen des Crocker-Grier Linking Algorithmus.}
    \label{fig:funkCrockGrierMod}
\end{figure}

Gerade bei hochdynamischen Ereignissen passiert es, dass in frühen Stufen nur wenige \gls{Assoziation}[en] getätigt werden können. Dadurch entstehen wieder mehr \gls{Assoziation}[sfehler]. Auch ist nicht mehr zu gewährleisten, dass der Rechenaufwand zu bewerkstelligen ist. Aus diesem Grund wird in solchen Situationen das adaptive Verfahren angewendet, das in der \gls{Bibliothek} \textit{Trackpy} \cite{Allan.2023} implementiert ist. Dadurch wird sichergestellt, dass die Netzwerke eine Maximalgröße nicht überschreiten und der Rechenaufwand zu bewältigen ist. \par

Ziel ist es, konstante \gls{Trajektorie}[n] zu generieren. Durch die Iterationen der Stufen können jedoch \gls{Fragmentation}[sfehler] provoziert werden. Das ist in der Abbildung \ref{fig:bspCrockGrierFrag} dargestellt. Unterbricht ein aktives Tier seine dynamische Bewegung für wenige \gls{Frame}[s] und verweilt auf einer Position, dann werden die \gls{Assoziation}[en] dieser wenigen \gls{Frame}[s] bereits in einer früheren Stufe getätigt. Die \gls{Detektion}[en] der dynamischen Bewegung werden erst in höheren Stufen \glsdisp{Assoziation}{assoziiert}. Eine konstante \gls{Trajektorie} ist dann nicht mehr möglich, da das Verweilen des Tieres bereits aus der weiteren Betrachtung herausgenommen wurde. 

\begin{figure}[htb]
    \centering
    \includegraphics[width=0.9\textwidth]{img/Grafiken/Fragmentation durch Stufen.png}
    \caption[Fragmentation durch die Modifikationen des Crocker-Grier Linking Algorithmus.]{Fragmentation durch die Modifikationen des Crocker-Grier Linking Algorithmus. Durch Veränderungen in der Geschwindigkeit werden zusammengehörende Detektionen in anderen Stufen assoziiert.}
    \label{fig:bspCrockGrierFrag}
\end{figure}

Ein Verifikationsmechanismus wurde eingeführt, welcher evaluiert, ob eine \gls{Trajektorie} in einer bestimmten Stufe plausibel ist oder nicht. Ist die \gls{Trajektorie} plausibel, gilt sie als verifiziert und wird in das Ergebnis des \gls{MOT} Systems aufgenommen. Kann die \gls{Trajektorie} nicht verifiziert werden, dann kommen ihrer \gls{Detektion}[en] in die nächste Stufe. \gls{Detektion}[en], die sich am Ende nicht in einer verifizierten \gls{Trajektorie} befinden, werden verworfen. Die Abbildung \ref{fig:TreeCrockGrierVerif} zeigt den Verifikationsmechanismus in Form eines Baumdiagramms. 

\begin{figure}[H]
    \centering
    \includegraphics[width=0.8\textwidth]{img/Grafiken/Baumdiagramm Verifizierung der Trajektorien.png}
    \caption[Baumdiagramm der Verifikation des modifizierten Crocker-Grier Algorithmus.]{Baumdiagramm des Verifikationsmechanismus des modifizierten Crocker-Grier Linking Algorithmus.}
    \label{fig:TreeCrockGrierVerif}
\end{figure}

Für die Verifikation wird das Video des Ereignisses mit herangezogen. Bei diesem wird ermittelt, wie sich die Pixel von \gls{Frame} zu \gls{Frame} verändern. Wenn \(\nommat{PIX}_t\) die Matrix aller Pixel in einem \gls{Frame} \(t\) ist, dann berechnet sich die Veränderung wie in \autoref{eq:Pixelveränderung} dargestellt.

\begin{equation}
\text{Pixelveränderung} = \frac{\sum_i \sum_j |pix_{t_{i,j}} - pix_{{t-1}_{i,j}}|}{\text{Pixelanzahl}}
\label{eq:Pixelveränderung}
\end{equation}

Dabei werden die Farbkanäle berücksichtigt. Ein \(Pixel \in \nommat{PIX}\) besitzt jeweils einen Kanal für Rot, Blau und Grün. Diese können die Werte \(\{0,1,2, \dots, 255\}\) annehmen. Mathematisch lässt sich jeder Pixel als Vektor \(\nomvec{pix}\) betrachten. Der Wertebereich der Pixelveränderung beträgt somit \([0, 765]\), da \(3 \cdot 255 = 765\). \par

Der Verlauf der Pixelveränderung wird für das Ereignis ermittelt. Der Veränderungswert dient als Aktivitätsindikator. Eine \gls{Trajektorie} muss nicht über das gesamte Ereignis verlaufen. Es wird der Zeitraum betrachtet, in dem die \gls{Trajektorie} aktiv war. Für diesen Zeitraum wird die mittlere Pixelveränderung berechnet und die Standardabweichung. Diese Werte dienen der Beurteilung, ob das Geschehen im Stall insgesamt dynamisch ist oder nicht. Bei einem dynamischen Geschehen, wie bei einem Kontrollgang, ist es sinnvoll, die \gls{Detektion}[en] bei einem größeren \(l\) zu betrachten, da \gls{Fragmentation}[sfehler] wahrscheinlicher sind. \par

Die mittlere Geschwindigkeit aller \gls{Trajektorie}[n] wird berechnet und die Geschwindigkeit der individuellen \gls{Trajektorie}[n] wird damit verglichen. Liegt die Geschwindigkeit signifikant höher als die mittlere Geschwindigkeit, wird das Potenzial für \gls{Fragmentation}[sfehler] als höher eingestuft und die \gls{Detektion}[en] werden in der nächsten Stufe erneut beurteilt. \par

Das letzte Verifikationskriterium ist die Schrittanzahl der \gls{Trajektorie}[n]. Von \gls{Trajektorie}[n], die über viele \gls{Frame}[s] verlaufen, ist tendenziell zu erwarten, dass sie konstanter sind. Somit fordert der Verifikationsmechanismus eine minimale \gls{Frame}[anzahl], bzw. Zeitschritten.\par

Die Verifikationskriterien der unterschiedlichen Stufen sind dynamisch. Da in höheren Stufen eine höhere Fehleranfälligkeit erwartet wird, sinkt die minimale \gls{Frame}[anzahl], um nicht zu viele \gls{Trajektorie}[n] auszusortieren. Auch die Toleranzschranken für die Dynamik steigen, da in den höheren Stufen mehr Dynamik erwartet wird. \par



\subsection{Methodisches Vorgehen der MOT Evaluationen}
Für die Evaluation stehen zwei \gls{Ground Truth} Ereignisse zur Verfügung: ein Kampfereignis und ein Kontrollgangereignis (\autoref{sec:Meth GT Quali}). Die Algorithmen, zwischen denen auszuwählen ist, sind der \acrshort{SORT} Algorithmus (\autoref{sec:MOT SORT}) und die modifizierte Variante des Crocker-Grier Linking Algorithmus (\autoref{sec:Meth CrockGrieMods}). Die fairste Bewertung ist mit der \gls{HOTA} Metrik (\autoref{sec:MOT HOTA}) zu erwarten, weshalb diese für die Evaluation verwendet wird. Ebenfalls wird dadurch eine genaue Fehleranalyse möglich. Die \textit{\gls{IDF1}-Metrik} und die \textit{\acrshort{CLEAR} \gls{MOT}} Metriken werden jedoch angegeben, da sie etablierter sind als \gls{HOTA}. Für die Berechnung der Metriken wird die Software \textit{TrackEval} benutzt \cite{TrackEval.2020}. Diese ist die offizielle Evaluationssoftware der \textit{MOTChallenge} und steht als Open-Source-Code zur Verfügung. Mit der Nutzung von \textit{TrackEval} wird sichergestellt, dass die Metriken korrekt berechnet werden. \par

Als Erstes wird für beide Algorithmen die Gesamtperformance bewertet. Dazu werden die beiden Algorithmen auf die Ereignisse angewendet, zu denen \gls{Ground Truth}[s] vorhanden sind. Für jeden \gls{Assoziation}[salgorithmus] ist anschließend ein Trackingergebnis vorhanden. Dieses beinhaltet die \gls{Trajektorie}[n]. Gespeichert werden die Trackingergebnisse als txt-Datei. Die Formatierung erfolgt angepasst an \textit{TrackEval}, gemäß \cite{MOT15}. Anschließend werden die Metriken berechnet. Wie \textit{TrackEval} auf eigene \gls{MOT} Systeme und mit eigenen \gls{Ground Truth}[s] anzuwenden ist, ist in der Dokumentation erläutert \cite{TrackEval.2020}. Die Evaluation der beiden Ereignisse wird gemittelt und über die Ergebnisse lassen sich die Algorithmen vergleichen. Neben den Wertungen der \gls{HOTA} Metrik erstellt \textit{TrackEval} einen Plot. Dieser zeigt den Verlauf der Wertungen über \(\alpha\). Die Abbildung \ref{fig:MPNTrack} zeigt einen solchen Plot. Er entstammt einem \gls{MOT} System namens \textit{MPNTrack}, welches bei der \textit{MOTChallenge} 2016 und 2017 angetreten ist. Das MOT-System ist hier als Beispiel angeführt. 

\begin{figure}[htb]
    \centering
    \includegraphics[width=0.9\textwidth, height=11cm]{img/Plots/MPNTrack MOT17.png}
    \caption[Beispiel eines Evaluationsergebnisses eines MOT Systems mit der HOTA Metrik.]{Beispiel eines Evaluationsergebnisses eines MOT Systems mit der HOTA Metrik.}
    \label{fig:MPNTrack}
\end{figure}

Während der Generation der Trajektorien werden die Laufzeiten der Algorithmen gemessen. Ebenfalls werden weitere Laufzeitmessungen durchgeführt, bei denen die Länge der Ereignisse stärker zu variieren ist. Dadurch wird ermöglicht, die Laufzeiten zu vergleichen und die zeitliche Skalierung in Bezug auf die Ereignislänge zu betraachten.\par 

Als Nächstes ist zu überprüfen, ob die \gls{Assoziation}[salgorithmen] alle Verhaltensweisen 
gleich gut \glsdisp{Assoziation}{assoziiert} bekommen. Dazu wird jedes Ereignis unabhängig voneinander evaluiert.\par
