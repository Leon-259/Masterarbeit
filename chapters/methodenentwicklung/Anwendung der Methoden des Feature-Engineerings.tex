\section{Anwendung der Methoden des Feature-Engineerings}
Über das Konzept in \ref{sec:Meth FeatExtr} \todo{Kapitelreferenzen beschriften} können Features aus den Rohdaten extrahiert werden. Im Vorfeld ist die Frage zu beantworten, welche Features extrahiert werden sollen. Diese Features werden in \ref{sec:Meth ExtrFeaturesVors} Vorgestellt. Mit den extrahierten Features als Grundlagen lassen sich weitere Features konstruieren (\ref{sec:ML FeatExtr}). Welche Methoden zu Feature-Konstruktion verwendet werden, wird in \ref{sec:Meth KonstrFeatures} erläutert.


\subsection{Vorstellung der extrahierten Features} \label{sec:Meth ExtrFeaturesVors}
Die charakteristischen Merkmale der Verhaltensweisen, sind in \ref{sec:Meth DefAufgabe} dargestellt. Diese waren teilweise bereits im Vorfeld bekannt und teilweise während der Verifizierung wahrgenommen. Dadurch ist Fachwissen vorhanden um maßgeschneiderte Features zu entwerfen. Fachwissen ist ein großer Vorteil für den Entwurf von Features, wie in \ref{sec:ML FeatExtr} erläutert. Es ist sinnvoll Bias einzuführen in Bezug auf die charakteristischen Merkmale. Features können so konstruiert werden, dass sie darauf ausgelegt sind bestimmte Aspekte der charakteristischen Merkmale zu erfassen. Ebenfalls ist durch das Fachwissen und das wissen, dass die Zusammensetzung der Ereignisse Bias behaftet ist (\ref{sec:Meth Datensatz}), besser abzuwägen, ob bestimmte Features den vorhandenen Bias verstärken. Auch können Features entworfen werden, welche gezielt versuchen den Bias abzuschwächen. \par

Das soll an einem Beispiel verdeutlicht werden. Ein charakteristischen Merkmal von Kontrollgängen sind prägnante Gruppenbewegungen. Die Tiere drängen kollektiv in Richtung des Tierhalters. Somit kann der Drift der Bewegung aller Tiere ein interessantes Feature sein. Die Driftrichtung würde jedoch den Bias in den Daten verstärken, da diese vermutlich sehr abhängig vom Stallbereich ist. Da die Ereignisse jedoch nur aus zwei Stallbereichen stammen, ist dies ein Feature was die Generalisierung über alle Stallbereiche beinträchtigen würde. Ein besseres Feature wäre der Betrag der Driftgeschwindigkeit. Dieser ist ein Indikator für eine starke kollektive Bewegung in eine unbestimmt Richtung und damit unabhäniger vom Stallbereich. \par 

Für die Feature-Erstellung ist nochmal darauf hinzuweisen, dass die Merkmale einer Sequenz, dem Ereignis, zu einem einzelnen Datenpunkt komprimiert werden. Die Features benutzen immer einen einzigen Wert, um eine Information aus mehreren Zeitschritten zusammenzufassen. Ebenfalls sind möglichst numerische Features zu erstellen, da gerade einfachere Modell mit diesen besser umgehen können, als mit Kategorischen Features. Im Anhang findet sich eine Tabelle mit ausführlichen Beschreibungen der Features, welche mit dem Konzept aus \ref{sec:Meth FeatExtr} für den Trainingsdatensatz extrahiert werden. insgesamt sind dies 40 Features. Diese werden im weiteren Verlauf Basis-Features genannt, da die weiteren Features aus diesen konsruiert werden.\par


\subsection{Konstruktion der Features} \label{sec:Meth KonstrFeatures}
Da eine ausführliche Analyse und Auswahl der Features geplant ist, wird bei der Konstruktion der Fokus auf Masse gelegt. Features werden Mittels allen Konstruktionsmethoden generiert, die in \ref{sec:ML FeatExtr} vorgestellt sind. Bei der Durchführung ist festzustellen, dass einige der Basis-Features keine Information beinhalten. Die minimale Entfernung und minimale Kurvigkeit sind in allen Ereignissen konstant null. Diese werden aus der Feature-Menge entfernt, somit reduziert sie die Anzahl der Basis-Features auf 38.\par

Zunächst werden Interaktionsfeatures erschaffen, indem die Features paarweise multipliziert werden, um eine neues Feature zu generieren. Auch die Multiplikation mit sich selbst ist darin eingeschlossen. Es entstehen 741 neue Features, wodurch die Featureanzahl auf 779 steigt. Mittels Gruppierung werden die 779 Features in Perzentile eingeteilt. Diese Features werden als Perzentil-Features bezeichnet. Ebenfalls werden die Power Transformationen auf die 779 Features angewendet. Die logarithmische Transformation wird mit der Basis 10 durchgeführt, da er vor allem die Größenordnungen stauchen soll. Auch die Box-Cox-Transformation wird auf alle 779 Features angewendet. \par

Bei der Berechnung der Perzentil-Features, ist die normierte maximale Schrittanzahl nicht gruppierbar, auch nicht in der Interaktion mit sich selbst. Das liegt an Invarianz in der Verteilung, da der Wert meisten 1 beträgt. Deshalb fallen diese beiden Features raus. 9 Weitere Features lassen sich aufgrund geringer Varianz nicht in Perzentile einteilen. Jedoch lassen sie sich in Dezentile einteilen. Da die Bezeichnung Perzentil-Features, somit faktisch nicht mehr ganz Korrekt ist, werden die als Quantil-Features bezeichnet. \par

Ziel der Gruppierungen und Power Transformationen ist es, jegliche Features, die Informationen beinhalten können unterschiedlich zu skalieren und zu formatieren. in der Featureauswahl soll dann Untersucht werden, welche Methode die Informationen des jeweiligen Features am besten verpackt. Die Tabelle \ref{tab:FeatureMenge} zeigt die Feature-Mengen nach der Transformation.


\begin{table}[ht]
    \centering
    \caption{Übersicht über die Feature-Mengen pro Kategorie.}
    \begin{tabular}{|l|r|}
        \hline
        Feature-Kategorie & Anzahl\\
        \hline
        Basis & 38 \\
        Interaktion & 741\\
        Quantile & 777\\
        Logarithmisch & 779\\
        Box-Cox & 779\\
        \hline
        \hline
        Gesamt & 3114\\
        \hline
    \end{tabular}
    \label{tab:FeatureMenge}
\end{table}