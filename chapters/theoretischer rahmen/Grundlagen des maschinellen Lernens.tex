\section{Grundlagen des maschinellen Lernens}
Traditionelle Computerprogramme bestehen aus einer Abfolge von Befehlen, welche dem Computer Schritt für Schritt erklären, was er tun soll. \glsdisp{ML}{Lern-Algorithmen} sind in der Lage eigenständig herauszufinden, was sie tun sollen. \Gls{ML} hat seine Anfänge in den 1950er Jahren. Seither wurde eine Vielzahl an Ansätzen entwickelt, um Maschinen Lernfähigkeit zu verleihen. Diese reichen von einfachen statistischen Modellen bis hin zu komplexen neuronalen Netzen. Heutzutage ist \gls{ML} fester Bestandteil unseres Alltags. Sei es durch Empfehlungsalgorithmen beim Online-Shopping, durch Spam-Filter für das E-Mail Postfach, in der Überwachung von öffentlichen Räumen durch \gls{MOT} oder durch den neusten Trend: Large Language Modells wie \textit{ChatGPT} \cite{Domingos.2015, Liu.2023}. Die Geschichte des \gls{ML} zeigt, dass für Computer oftmals die Aufgaben am herausforderndsten sind, welche Menschen intuitiv bewältigt werden können. Es ist schwierig einem Computer den Lösungsweg für solche Aufgaben zu erklären \cite{Goodfellow.2016}. \par

In \cite{Mitchell.1997} wird \gls{ML} beschrieben, als ein Lernprozess, welcher ein Computer automatisch durchführt. Es wird wie folgt definiert: Ein Computerprogramm ist Lernfähig, wenn es sich durch Erfahrung \(E\) in seiner Performance \(P\) im Bezug auf die Bewältigung einer Aufgabe \(A\) verbessert. In einem Beispiel möchte ein Autohändler den Verkaufswert von Autos schätzen. Dazu muss er zunächst auswählen, welche Eigenschaften eines Autos er für die Schätzung nutzen möchte. Er entscheidet sich für die Anzahl der gefahrenen Kilometer und das Alter des Autos. Solche Eigenschaften werden \gls{Feature}[s] genannt. \gls{Feature}[s] sind Merkmale, welche Informationen zur Bewältigung der Aufgabe \(A\) beisteuern. Die \gls{Feature}[s] werden in einer \gls{Datenmatrix} \(\nommat{X}\) angeordnet. Bezogen auf das Beispiel beinhalten die Spalten in \(\nommat{X}\) die Werte der gefahrenen Kilometer und des Alters der Autos. Die Zeilen von \(X\) beinhalten die \gls{Feature}[s] zu jeweils einem Auto. Neben der \gls{Datenmatrix} benötigen viele Modelle noch einen \gls{Zielvektor} \(\nomvec{y}\). Dieser Beinhaltet die Ziel-Werte für die Aufgabe A. Im Beispiel ist jedes Element \(\nomvec{y}_i\) der Wert eines Autos. Jede Reihe \(\nommat{X}_{i,:}\) beinhaltet die Informationen zu der Anzahl der gefahrenen Kilometer und des Alters von einem spezifischen Auto. Das Element \(\nomvec{y}_i\) ist der Wert für den dieses spezifische Auto verkauft wurde. Zusammen bilden \(\nommat{X}\) und \(\nomvec{y}\) einen Datensatz. Dieser Datensatz ist die Erfahrung \(E\) mit dessen Hilfe das Programm lernen soll. Hat der Autohändler \(N=20\) Erfahrungswerte so ist der Datensatz Aufgebaut aus \(N\) Datenpunkten \(D = \{(\nommat{X}_{i,:}, \nomvec{y}_i)\}_{i}^{N}\) \cite{Goodfellow.2016, Burkov.2019, ShalevShwartz.2014}. Die Tabelle \ref{tab:BspMLAuto} zeigt den Aufbau des Datensatzes aus dem Beispiel. 


\begin{table}
    \centering
    \begin{tabular}{|r|r|r|}
     \hline
        Verkaufswert in €   & gefahrene Kilometer   & Alter in Jahren\\
     \hline
        10200               & 52000                 & 5             \\
     \hline
        7100                & 140000                & 12            \\
     \hline
        \vdots              & \vdots                & \vdots        \\
     \hline
         23800              & 25000                 & 2             \\
      \hline
    \end{tabular}
    \caption{Aufbau des Datensatzes für das Beispiel der Schätzung von Verkaufswerten von Autos. Die Spalte \textit{Verkaufswerte} entspricht \(\nomvec{y}\). Die Spalten \textit{gefahrene Kilometer} und \textit{Alter in Jahren} bilden \(\nommat{X}\). }
    \label{tab:BspMLAuto}
\end{table}

Damit ist die Aufgabe \(A\) und die Erfahrung \(E\) für das Beispiel definiert. 

\begin{itemize}
    \item A: Schätzung des Verkaufswerts von Autos.
    \item E: Datensatz aus den \gls{Feature}[s] und den \glsdisp{Zielvektor}{Zielwerten}.
\end{itemize}

\emptyFigure{Beispiel einer \glsdisp{ML}{maschinellen Lernaufgabe}. Schätzung vom Verkaufswerten von Autos. Auf der y-Achse sind ist der \glsdisp{Zielvektor}{Zielwert} dargestellt. Auf der y-Achse ist das \gls{Feature} \textit{Anzahl der gefahrenen Kilometer} abgebildet. Die Werte der Achsen sind Normiert.}{fig:BspMLAuto}

Die Abbildung \ref{fig:BspMLAuto} zeigt die die Punkte in \(D\) für den \glsdisp{Zielvektor}{Zielwert} und das \gls{Feature} \textit{Anzahl der gefahrenen Kilometer}. Um ein Lernfähiges Programm zu erhalten muss eine mathematische Repräsentation für die Bewältigung der Aufgabe \(A\) gefunden werden. Durch diese mathematische Repräsentation, ist die Performance \(P\) des Programms bewertbar \cite{Mitchell.1997}. Für das Beispiel des Autohändlers wird eine Lineare Regression durchgeführt. 