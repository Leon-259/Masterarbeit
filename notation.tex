%\newcommand{\nomvec}[1]{\vec{#1}} % Spaltenvektor
%\newcommand{\nomvecT}[1]{\vec{#1}^\intercal} % Zeilenvektor
%\newcommand{\nommat}[1]{\boldsymbol{#1}} % Matrix

\nomenclature[01]{\(\nomvec{a}\)}{Ein Spaltenvektor}
\nomenclature[02]{\(\nomvec{a}_i\)}{Element i eines Vektors \(\nomvec{a}\)}
\nomenclature[03]{\(\nomvecT{a}\)}{Ein Zeilenvektor, Transponierter Vektor}
\nomenclature[04]{\(\nommat{A}\)}{Eine Matrix}
\nomenclature[05]{\(a_{i,j}\)}{Element in Reihe i und Spalte j einer Matrix \(\nommat{A}\)}
\nomenclature[06]{\(\nommat{A}_{i,:}\)}{Reihe i einer Matrix \(\nommat{A}\)}
\nomenclature[07]{\(\nommat{A}_{:,j}\)}{Spalte j einer Matrix\( \nommat{A}\)}
\nomenclature[08]{\(A\)}{Eine Menge}
\nomenclature[09]{\(\lvert A \lvert\)}{Mächtigkeit einer Menge}
\nomenclature[09]{\(\emptyset\)}{Die leere Menge}


