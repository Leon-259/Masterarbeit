\usepackage{graphicx} % Required for inserting images
\usepackage[T1]{fontenc} %Notwendig um Umlaute benutzen zu können 
\usepackage[ngerman]{babel} %import vom deutschen Sprachpaket
\usepackage{microtype} %Kleine Variationen im Textabstand um Silbentrennung zu reduzieren
\usepackage{lmodern}
\usepackage{hyphenat} %Für manuelle Silbentrennung
\usepackage[headsepline, plainheadsepline, markcase=used]{scrlayer-scrpage}     %Seitendesign einstellungen
\usepackage{ziffer} %deutsche dezimaltrennung als ,
\usepackage{listings} %Für Code auszüge
\usepackage{xcolor} %Für Farbgebungen
\usepackage{scrhack}
\usepackage{hyperref}
\usepackage{caption}
\usepackage{bookmark}
\usepackage{amsmath}
\usepackage{geometry}
\usepackage{blindtext}
\usepackage{layout}

\captionsetup{skip=10pt}

\renewcommand{\familydefault}{\sfdefault}       %Font übernehmen für das ganze Dokument

\graphicspath{{img/}} % Ordner für Bilder Festlegen
\bibliographystyle{ieeetr}

\pagestyle{scrheadings} %Seitenstil preset
\clearpairofpagestyles  %Seitenzahlen aus der Fußzeile löschen
\chead{\headmark}       %Kapitelüberschriften außen in der Kopfzeile
\ohead*[\pagemark]{\pagemark}   

\geometry{
  inner=2.5cm, % Größerer innerer Rand für die Bindung
  outer=1.5cm, % Kleinerer äußerer Rand
  top=1cm,
  bottom=3cm,
  headsep=2pt,
  includehead
}

% Umgebungsfarben für Codeausschnitte
\definecolor{codegreen}{rgb}{0,0.6,0}
\definecolor{codegray}{rgb}{0.5,0.5,0.5}
\definecolor{codepurple}{rgb}{0.58,0,0.82}
\definecolor{backcolour}{rgb}{0.95,0.95,0.92}
\lstdefinestyle{mystyle}{
    backgroundcolor=\color{backcolour},   
    commentstyle=\color{codegreen},
    keywordstyle=\color{magenta},
    numberstyle=\tiny\color{codegray},
    stringstyle=\color{codepurple},
    basicstyle=\ttfamily\footnotesize,
    breakatwhitespace=false,         
    breaklines=true,                 
    captionpos=b,                    
    keepspaces=true,                 
    numbers=left,                    
    numbersep=5pt,                  
    showspaces=false,                
    showstringspaces=false,
    showtabs=false,                  
    tabsize=2
}

\lstset{style=mystyle}

%Neudefinition des \mainmatter befehls, um bei Aufruf die letze Seitenzahl des 
%Frontmatters zu speichern, um beim Backmatter wieder drauf zurückgreifen zu können
\makeatletter
\newcounter{savedfrontmatterpage}
\renewcommand{\mainmatter}{%
  \cleardoublepage
  \setcounter{savedfrontmatterpage}{\value{page}}%
  \@mainmattertrue
  \pagenumbering{arabic}%
  \clearpairofpagestyles
  \chead{\headmark}
  \ofoot*[\pagemark]{\pagemark}
}

%Neudefiniton des backmatter-Befehls, um bei AUfruf die Seitenzahl auf die letzte Seitenzahl des Frontmatters zurückzugreifen
\renewcommand\backmatter{%
  % Prüft, ob die Seite ungerade ist (rechte Seite)
  \ifodd\value{page}%
    \cleardoublepage % Wechselt zur nächsten rechten Seite, wenn aktuell auf einer rechten Seite
  \else
    \clearpage % Nur ein einfacher Seitenumbruch, wenn auf einer linken Seite
  \fi
  \@mainmatterfalse
  \pagenumbering{Roman}%
  \setcounter{page}{\value{savedfrontmatterpage}}%
  \clearpairofpagestyles
  \chead{\headmark}                                                               %Kapitelüberschriften außen in der Kopfzeile
  \ohead*[\pagemark]{\pagemark}                                                    %Seitenzahlen innen in der Kopfzeile
}
\makeatother

\newcommand{\theauthor}{Leon Grude}
\newcommand{\thetitle}{Entwicklung eines Moduls zur automatischen Klassifikation von Verhaltensweisen von Mastputen mittels Machine-Learning-Methoden}
\newcommand{\thedate}{02.02.2024}