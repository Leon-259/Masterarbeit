\usepackage{graphicx} % Required for inserting images
\usepackage{float}
\usepackage[T1]{fontenc} %Notwendig um Umlaute benutzen zu können 
\usepackage[ngerman]{babel} %import vom deutschen Sprachpaket
\usepackage{microtype} %Kleine Variationen im Textabstand um Silbentrennung zu reduzieren
\usepackage{lmodern}
\usepackage{hyphenat} %Für manuelle Silbentrennung
\usepackage[headsepline, plainheadsepline, markcase=used]{scrlayer-scrpage}     %Seitendesign einstellungen
\usepackage{ziffer} %deutsche dezimaltrennung als ,
\usepackage{listings} %Für Code auszüge
\usepackage{xcolor} %Für Farbgebungen
\usepackage{scrhack}
\usepackage{hyperref}
\usepackage{caption}
\usepackage{subcaption}
\usepackage{bookmark}
\usepackage{amsmath}
\usepackage{amssymb}
\usepackage{geometry}
\usepackage{blindtext}
\usepackage[acronym]{glossaries} %Erstellen eines glossar
\usepackage[backend=biber, sortcites, citestyle=numeric]{biblatex}
\usepackage[autostyle]{csquotes}
\usepackage{todonotes} %setzen von todo{} notizen. Liste am ende des dokuments wird erstellt
\usepackage{nomencl} %Erstellen eines Verzeichnisses für die Mathematische Notation
\usepackage{booktabs} % Für schönere Tabellen
\usepackage{siunitx} % Ermöglicht die Ausrichtung von Zahlen in der Tabelle
\usepackage{adjustbox} % Für die Ausrichtung der rotierten Boxen
\usepackage{longtable} % Lange Tabellen über mehrere Seiten erstellen 
\usepackage{array}
\usepackage[most]{tcolorbox}
\usepackage{float}
\usepackage{wrapfig}
\usepackage{multirow}
\usepackage{rotating}
\usepackage{makecell}

\restylefloat{figure}
\captionsetup{format=plain,indention=0cm}

\addbibresource{bib/bib.bib}

\captionsetup{skip=10pt}

\renewcommand{\familydefault}{\sfdefault}       %Font übernehmen für das ganze Dokument

\graphicspath{{img/}} % Ordner für Bilder Festlegen

%\bibliographystyle{ieeetr}

\makeglossaries


\renewcommand\theadalign{bc} % Zentrierung im Spaltenkopf
\renewcommand\theadfont{\bfseries} % Fette Schrift im Spaltenkopf
\renewcommand\theadgape{\Gape[4pt]} % Abstand nach oben im Spaltenkopf
\newcommand\rot{\rotatebox{90}} % Kurzbefehl für das Rotieren


\makenomenclature
\renewcommand{\nomname}{Mathematische Notation}
\setlength{\nomlabelwidth}{3cm}
% Befehle für unterschiedliche Formatierungen
%\newcommand{\nomupbold}[1]{\mathrm{\mathbf{#1}}} % Tensoren fett und gerade
\newcommand{\nomvec}[1]{\vec{#1}} % Spaltenvektor
\newcommand{\nomvecT}[1]{\vec{#1}^\mathrm{T}} % Zeilenvektor
\newcommand{\nommat}[1]{\boldsymbol{#1}} % Matrix


\pagestyle{scrheadings} %Seitenstil preset
\clearpairofpagestyles  %Seitenzahlen aus der Fußzeile löschen
\chead{\headmark}       %Kapitelüberschriften außen in der Kopfzeile
\ohead*[\pagemark]{\pagemark}   

\geometry{
  inner=2.5cm, % Größerer innerer Rand für die Bindung
  outer=1.5cm, % Kleinerer äußerer Rand
  top=0.6cm,
  bottom=3cm,
  headsep=15pt,
  includehead
}

\sisetup{
  output-decimal-marker = {,}, % Komma als Dezimaltrennzeichen
  table-format = 2.2, % Zwei Ziffern vor und zwei nach dem Dezimaltrennzeichen
  table-space-text-post = {,} % Reserve space for the comma in decimals
}


% Umgebungsfarben für Codeausschnitte
\definecolor{codegreen}{rgb}{0,0.6,0}
\definecolor{codegray}{rgb}{0.5,0.5,0.5}
\definecolor{codepurple}{rgb}{0.58,0,0.82}
\definecolor{backcolour}{rgb}{0.97,0.97,1}
\lstdefinestyle{mystyle}{
    backgroundcolor=\color{backcolour},   
    commentstyle=\color{codegreen},
    keywordstyle=\color{magenta},
    numberstyle=\tiny\color{codegray},
    stringstyle=\color{codepurple},
    basicstyle=\ttfamily,
    breakatwhitespace=false,         
    breaklines=true,                 
    captionpos=b,                    
    keepspaces=true,                 
    %numbers=left,                    
    numbersep=5pt,                  
    showspaces=false,                
    showstringspaces=false,
    showtabs=false,                  
    tabsize=2
}

\lstset{style=mystyle}
\lstset{literate=%
  {Ö}{{\"O}}1
  {Ä}{{\"A}}1
  {Ü}{{\"U}}1
  {ß}{{\ss}}1
  {ü}{{\"u}}1
  {ä}{{\"a}}1
  {ö}{{\"o}}1
}

\addto\captionsngerman{%
  \renewcommand{\lstlistlistingname}{Codeauszugsverzeichnis}
}
\addto\captionsngerman{
  \renewcommand{\lstlistingname}{Codeauszug}
}

%Neudefinition des \mainmatter befehls, um bei Aufruf die letze Seitenzahl des 
%Frontmatters zu speichern, um beim Backmatter wieder drauf zurückgreifen zu können
\makeatletter
\newcounter{savedfrontmatterpage}
\renewcommand{\mainmatter}{%
  \cleardoublepage
  \setcounter{savedfrontmatterpage}{\value{page}}%
  \@mainmattertrue
  \pagenumbering{arabic}%
  \clearpairofpagestyles
  \ohead{\headmark}
  \ofoot*[\pagemark]{\pagemark}
}

%Neudefiniton des backmatter-Befehls, um bei AUfruf die Seitenzahl auf die letzte Seitenzahl des Frontmatters zurückzugreifen
\renewcommand\backmatter{%
  % Prüft, ob die Seite ungerade ist (rechte Seite)
  \ifodd\value{page}%
    \cleardoublepage % Wechselt zur nächsten rechten Seite, wenn aktuell auf einer rechten Seite
  \else
    \clearpage % Nur ein einfacher Seitenumbruch, wenn auf einer linken Seite
  \fi
  \@mainmatterfalse
  \pagenumbering{Roman}%
  \setcounter{page}{\value{savedfrontmatterpage}}%
  \clearpairofpagestyles
  \chead{\headmark}                                                               %Kapitelüberschriften außen in der Kopfzeile
  \ohead*[\pagemark]{\pagemark}                                                    %Seitenzahlen innen in der Kopfzeile
}
\makeatother

\newcommand{\theauthor}{Leon Grude}
\newcommand{\thetitle}{Entwicklung eines Moduls zur automatischen Klassifikation von Verhaltensweisen von Mastputen mittels Machine-Learning-Methoden}
\newcommand{\thedate}{11.03.2024}


% Command to create a glossary entry with correspondent acronym.
% Args : 1: acronym/name, 2: long name, 3: description
\newcommand{\newglossaryentrywithacronym}[3]{
    %%% The glossary entry the acronym links to   
    \newglossaryentry{#1_gls}{
        name={#1},
        long={#2},
        description={#3}
    }

    % Acronym pointing to glossary
    \newglossaryentry{#1}{
        type=\acronymtype,
        name={#1},
        description={#2},
        first={#2 (#1)\glsadd{#1_gls}},
        see=[Glossar:]{#1_gls}
    }
}

%gerade Inizes
\makeatletter
\begingroup
\catcode`\_=\active
\protected\gdef_{\@ifnextchar|\subtextit\subtextup }
\endgroup
\def\subtextit|#1|{\sb{#1}}
\def\subtextup#1{\sb{\mathrm{#1}}}
\AtBeginDocument{\catcode`\_=12 \mathcode`\_=32768}
\makeatother

\newcommand{\dubpar}{\vspace{0.6cm}}

\newcommand{\gfuss}[1]{\glqq#1\grqq}

\lstnewenvironment{pythoncode}[3][]{
  \lstset{
    language=Python,
    caption=#2,
    captionpos=b,
    label=#3,
    #1
  }
}{}

\lstnewenvironment{pythoncodeAnhang}[2][]{
  \lstset{
    language=Python,
    caption=#2,
    captionpos=b,
    #1
  }
}{}