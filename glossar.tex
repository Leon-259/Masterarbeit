%%%%%%%%%%%%%%%%%%%%%%%%%%%%%%%%%%%%%%%%%%%%%%%%%%%%%%%%%%%%%%%%%%%%
%-----------------------Allgemeines---------------------------------
%%%%%%%%%%%%%%%%%%%%%%%%%%%%%%%%%%%%%%%%%%%%%%%%%%%%%%%%%%%%%%%%%%%%

\newglossaryentry{Modul}
{
        name=Modul,
        description={In der Informationstechnik ist ein Modul ein Baustein eines Systems, welcher sich einfach austauschen lässt. Es kapselt Funktinoalität in einem eigenständigen Block. Module sind Systemelemente, welche in sich selbst Systeme sind}
}

\newglossaryentry{System}
{
        name=System,
        description={Eine eindeutig abgegrenzte Einheit bestehend aus verschiedenen Subsystemen und Systemelementen sowie deren Verknüpfungen}
}

\newglossaryentry{Frame}
{
        name=Frame,
        description={Im Kontext von Videos bezieht sich Frame auf ein einzelnes Bild in einer Reihe von Bildern, die zusammen ein Video bilden.}
}

\newglossaryentry{Mid-Level Aufgabe}
{
        name=Mid-Level Aufgabe,
        description={Mid-Level Aufgaben sind Tätigkeiten oder Prozesse, die zwischen Low-Level- und High-Level-Aufgaben notwendig sind. Mid-Level-Aufgaben fokussieren sich auf die Integration und Koordination zwischen diesen beiden Ebenen, um eine effiziente Ausführung von Anwendungen zu gewährleisten.}
}

\newglossaryentry{Ereignis}
{
        name=Ereignis,
        description={Im Kontext dieser Arbeit bezieht sich ein Ereignis auf eine spezifische Verhaltensweise, die innerhalb eines definierten Zeitraums auftritt.}
}

\newglossaryentry{Brownsche Bewegung}
{
        name=Brownsche Bewegung,
        description={Ist eine zufällige Bewegung von Teilchen in einer Flüssigkeit oder in einem Gas, die auf Stöße mit den Molekülen des Mediums zurückzuführen ist. Die Bewegungen sind zufällig.}
}

\newglossaryentrywithacronym
{IoU}
{Intersection over Union}
{Das Verhältnis des Überlappungsbereichs zwischen der vorhergesagten Bounding-Box und der wahren Bounding-Box zum Gesamtvereinigungsbereich beider Boxen.}

\newglossaryentry{Just-in-time}
{
        name=Just-in-time Kompilierung,
        description={Bei einer Kompilierung wird ein Computerprogramm in maschinenlesbaren Code übersetzt. Dies geschieht üblicherweise vor der Programmausführung. Bei Just-in-time Kompilierung erfolgt die Kompilierung während der Ausführung, in dem Moment wo der zu kompilierende Programmabschnitt aufgerufen wird.}
}

\newglossaryentry{Overhead}
{
        name=Overhead,
        description={Bezeichnet zusätzlichen Rechenaufwand, der durch die Ausführung einer bestimmten Aufgabe entsteht, welche nicht direkt zur eigentlichen Funktionalität beiträgt.}
}


\newglossaryentry{Echtzeitfähigkeit}
{
        name=Echtzeitfähigkeit,
        description={Echtzeitfähigkeit bedeutet, dass ein System innerhalb einer festgelegten Zeitspanne auf den Eingang reagiert. Das Ergebnis muss nach einer vorgegebenen Dauer feststehen \cite{Scholz.2005}.}
}

%%%%%%%%%%%%%%%%%%%%%%%%%%%%%%%%%%%%%%%%%%%%%%%%%%%%%%%%%%%%%%%%%%%%
%-----------------------MOT Allgemein-----------------------------------------
%%%%%%%%%%%%%%%%%%%%%%%%%%%%%%%%%%%%%%%%%%%%%%%%%%%%%%%%%%%%%%%%%%%%

\newglossaryentrywithacronym
{MOT}
{Multi-Object Tracking}
{Ein Teilbereich des maschinellen Sehens, das darauf abzielt, mehrere Objekte in einem Video zu detektieren, lokalisieren und über die Zeit so zu assoziieren, dass die Identitäten konstant zugeordnet bleiben \cite{HOTA}.}

\newglossaryentry{Tracking}
{
name=Tracking,
description={Die Verfolgung und der Bewegung von Objekten über die Zeit.}
}

\newglossaryentry{Detektionsbasiertes Tracking}
{
name=Detektionsbasiertes Tracking,
description={Ein Ansatz im Multi-Object Tracking, bei dem Objekte zunächst in jedem Frame detektiert und anschließend über die Zeit hinweg verfolgt werden.}
}

\newglossaryentry{Detektion}
{
name=Detektion,
description={Das Erkennen und Markieren von Objekten in Frames, vorallem im Kontext des Multi-Objekt Trackings.}
}

\newglossaryentry{Assoziation}
{
name=Assoziation,
description={Die Zuordnung von erkannten Objekten über aufeinanderfolgende Frames, um die  Identitäten der Objekte konstant zu halten. Vorallem im Kontext des Multi-Objekt Trackings verwendet.}
}

\newglossaryentry{Lokalisation}
{
name=Lokalisation,
description={Die Bestimmung der genauen Position eines Objekts innerhalb eines Frames, oft ausgedrückt durch Koordinaten oder eine Bounding Box. Vorallem im Kontext des Multi-Objekt Trackings verwendet.}
}

\newglossaryentry{Offline Tracking}
{
name=Offline Tracking,
description={Ein Tracking-Verfahren, bei dem alle Daten vor der Verarbeitung zur Verfügung stehen.}
}

\newglossaryentry{Online Tracking}
{
name=Online Tracking,
description={Ein Tracking-Verfahren, das die Daten sequenziell verarbeitet, ohne auf zukünftige Frames zuzugreifen. Ideal für Echtzeitanwendungen.}
}

\newglossaryentry{Trajektorie}
{
name=Trajektorie,
description={Die Bewegungslinie, die ein Objekt über die Zeit in einem Raum durchläuft. Eine Trajektorie ist eine Sequenz von Positionen des Objekts.}
}

\newglossaryentry{Bounding Box}
{
name=Bounding Box,
description={Ein rechteckiger Rahmen, der verwendet wird, um die Position und die Größe eines Objekts zu definieren.}
}

\newglossaryentry{Ground Truth}
{
name=Ground Truth,
description={Die genauen Detektionen, Positionen und Assoziationen der Objekte in einem Ereignis. Die Ground Truth dient als Refernzen für die Evalutaion von Multi-Object Tracking Systemen.}
}


%%%%%%%%%%%%%%%%%%%%%%%%%%%%%%%%%%%%%%%%%%%%%%%%%%%%%%%%%%%%%%%%%%%%
%-----------------------MOT Fehler-----------------------------------------
%%%%%%%%%%%%%%%%%%%%%%%%%%%%%%%%%%%%%%%%%%%%%%%%%%%%%%%%%%%%%%%%%%%%


%%%%%%%%%%%%%%%%%%%%%%%%%%%%%%%%%%%%%%%%%%%%%%%%%%%%%%%%%%%%%%
%%%%%%%%%%%%%%%%%%%%%%%%%%%%%%%%%%%%%%%%%%%%%%%%%%%%%%%%%%%%%%
%%%%%%%%%%%%%%%%%%%%%%%%%%%%%%%%%%%%%%%%%%%%%%%%%%%%%%%%%%%%%%
%%%%%%%%%%%%%%%%%%%%%%%%%%%%%%%%%%%%%%%%%%%%%%%%%%%%%%%%%%%%%%
\newglossaryentrywithacronym
{EP}
{Echt positive Detektion}
{Es ist eine korrekte Detektion. Eine echt positive Detektion tritt auf, wenn ein MOT System ein Objekt detektiert, welches wirklich existiert.}

\newglossaryentrywithacronym
{FP}
{Falsch positive Detektion}
{Ein Detektionsfehler. Eine falsch positive Detektion tritt auf, wenn ein MOT System ein Objekt detektiert, welches in Wirklichkeit nicht da ist.}

\newglossaryentrywithacronym
{FN}
{Falsch negative Detektion}
{Ein Detektionsfehler. Eine falsch negative Detektion tritt auf, wenn ein MOT System ein Objekt, welches in Wirklichkeit da ist, nich detektiert.}

\newglossaryentry{Fragmentation}
{
name=Fragmentation,
description={Ein Assoziationsfehler. Eine Fragmentation tritt auf, wenn ein Objekt in den vergangenen Frames eine ID \(a\) zugewiesen bekommen hat und im aktuellen Frame eine ID \(b\) erhält. Die korrekte Trajektorie des Objektes besteht aus dem Fragment mit der ID \(a\) und \(b\). Eine unvollständige Trajektorie zählt ebenfalls als Fragmentation.}
}

\newglossaryentry{Merging Fehler}
{
name=Merging Fehler,
description={Ein Assoziationsfehler. Ein Merging Fehler ist das Gegenstück zur Fragmentation. Er tritt auf, wenn zwei Objekte, Objekt \(a\) und Objekt \(b\) vom System die gleiche ID erhalten. Das System hält die beiden Objekte für ein einzelnes Objekt.}
}

\newglossaryentry{Lokalisationsfehler}
{
name=Lokalisationsfehler,
description={Eine Abweichung von der genauen Position des detektierten Objekts.}
}

\newglossaryentrywithacronym
{IDSW}
{Identity Switch}
{Fehler eines MOT Systems. Ein Identity Switch tritt auf, wenn sich die Identifikationsnummer (ID) eines Objektes ändert, obwohl sie konstant bleiben sollte. Dabei berücksichtigt eine Identity Switch jedoch keine unvollständigen Trajektorie. Wird eine Trajektorie
abgebrochen, obwohl sich das Objekt weiter im Ereignis befindet, gibt es keine ID die sich
ändern kann.}

%%%%%%%%%%%%%%%%%%%%%%%%%%%%%%%%%%%%%%%%%%%%%%%%%%%%%%%%%%%%%%%%%%%%
%-----------------------MOT Metriken-----------------------------------------
%%%%%%%%%%%%%%%%%%%%%%%%%%%%%%%%%%%%%%%%%%%%%%%%%%%%%%%%%%%%%%%%%%%%

\newglossaryentrywithacronym
{MOTP}
{Multiple Object Tracking Precision}
{Teil der \textit{\acrshort{CLEAR} \gls{MOT}} Metriken. Es ist ein Maß für die Präzision, mit der Lokalisation eines MOT Systems. MOTP berechnet sich über den Mittelwert der \(IoU\) aller echt positiven Detektionen.}

\newglossaryentrywithacronym
{MOTA}
{Multiple Object Tracking Accuracy}
{Teil der \textit{\acrshort{CLEAR} \gls{MOT}} Metriken. Es ist ein Maß für die Gesamtgenauigkeit eines MOT-Systems, das die Anzahl der echt positiven Detektionen, die falsch negativen Detektionen und die IDSWs berücksichtigt.}

\newglossaryentrywithacronym
{MODA}
{Multiple Object Detection Accuracy}
{Teil der \textit{\acrshort{CLEAR} \gls{MOT}} Metriken. Es ist ein Maß für die Genauigkeit der Detektion. Die Berechnung erfolgt wie bei MOTA, nur ohne die Berücksichtigung der IDSWs.}

\newglossaryentry{IDF1}
{
name=IDF1,
description={Es ist ein Maß für die Gesamtgenauigkeit eines MOT-Systems. Sie misst das Verhältnis der korrekt assoziierten Detektionen zur Gesamtzahl der System-Detektionen und der Anzahl der Ground Truth Detektionen.}
}

\newglossaryentrywithacronym
{HOTA}
{Higher Order Tracking Accuracy}
{Eine fortgeschrittene Metrik für die Bewertung von MOT Systemen, die sowohl die Detektions- als auch die Assoziationsgenauigkeit berücksichtigt. HOTA ist darauf ausgerichtet, die Teilkomponenten eines MOT Systems fair zu bewerten. Es ermöglicht eine ausführliche Systemanalyse mittels Submetriken.}

\newglossaryentrywithacronym
{DetA}
{Detection Accuracy}
{Submetrik der HOTA Metrik. Es ist ein Maß für die Gesamtgenauigkeit der Detektion. DetA berücksichtigt den Einfluss von unterschiedlichen Detektionsfehlern gleichwertig.}

\newglossaryentrywithacronym
{AssA}
{Association Accuracy}
{Submetrik der HOTA Metrik. Es ist ein Maß für die Gesamtgenauigkeit der Assoziation. AssA berücksichtigt den Einfluss von unterschiedlichen Assoziationsfehlern gleichwertig.}

\newglossaryentrywithacronym
{DetRe}
{Detection Recall}
{Submetrik der HOTA Metrik. Es ist ein Maß für die Fehleranfälligkeit des MOT Systems für falsch negative Detektionen.}

\newglossaryentrywithacronym
{DetPr}
{Detection Precision}
{Submetrik der HOTA Metrik. Es ist ein Maß für die Fehleranfälligkeit des MOT Systems für falsch positive Detektionen.}

\newglossaryentrywithacronym
{AssRe}
{Association Recall}
{Submetrik der HOTA Metrik. Es ist ein Maß für die Fehleranfälligkeit des MOT Systems für falsch negative Assoziationen. AssRe erfasst die Häufung von Fragementationsfehlern.}

\newglossaryentrywithacronym
{AssPr}
{Association Precision}
{Submetrik der HOTA Metrik. Es ist ein Maß für die Fehleranfälligkeit des MOT Systems für falsch positive Assoziationen. AssPr erfasst die Häufung von Merging Fehlern.}

\newglossaryentrywithacronym
{LocA}
{Localization Accuracy}
{Submetrik der HOTA Metrik. Es ist ein Maß für die Gesamtgenauigkeit der Lokalisation eines MOT Systems.}

\newglossaryentrywithacronym
{EPA}
{Echt positive Assoziation}
{Teil des Assoziationskonzepts, welches die HOTA Metrik verwendet, zur Assoziationsevaluation. Die echt positiven Assoziationen \(EPA(e)\) einer echt positiven Detektion \(e\) sind alle EP , welche die gleiche ID wie \(e\) besitzen.}

\newglossaryentrywithacronym
{FPA}
{Falsch positive Assoziation}
{Teil des Assoziationskonzepts, welches die HOTA Metrik verwendet, zur Assoziationsevaluation. Die falsch positiven Assoziationen \(FPA(e)\) einer echt positiven Detektion \(e\) sind alle System-Detektionen, welche die gleiche ID wie \(e\) besitzen, wo jedoch die Ground Truth ID eine andere ist.}

\newglossaryentrywithacronym
{FNA}
{Falsch negative Assoziation}
{Teil des Assoziationskonzepts, welches die HOTA Metrik verwendet, zur Assoziationsevaluation. Die falsch negativen Assoziationen \(FNA(e)\) einer echt positiven Detektion \(e\) sind alle  Ground-Truth-Detektionen, welche die gleiche ID wie \(e\) besitzen, wo jedoch die ID der System-Detektionen eine andere ist.}

%%%%%%%%%%%%%%%%%%%%%%%%%%%%%%%%%%%%%%%%%%%%%%%%%%%%%%%%%%%%%%%%%%%%
%------------------Machine Learning---------------------------------
%%%%%%%%%%%%%%%%%%%%%%%%%%%%%%%%%%%%%%%%%%%%%%%%%%%%%%%%%%%%%%%%%%%%

\newglossaryentry{ML}
{
        name=Maschinelles Lernen,
        text=maschinelles Lernen,
        description={Ein Bereich der künstlichen Intelligenz. Lern-Algorithmen sind in der Lage eigenständig herauszufinden, was sie tun sollen. Ein Computerprogramm ist lernfähig, wenn es sich durch Erfahrung \(E\) in seiner Performance \(P\) im Bezug auf die Bewältigung einer Aufgabe \(A\) verbessert.}
} 

\newglossaryentry{Feature}
{
        name=Feature,
        description={Features sind die Erfahrung \(E\), mit welcher Lern-Algorithmen herausfinden was sie tun sollen. Es sind Merkmale, welche Informationen zu der zu lernenden Aufgabe \(A\) besitzen. I.d.R drücken Features ihre Information numerisch aus. Kategorische Features sind jedoch ebenfalls üblich.}
}

\newglossaryentry{Datenmatrix}
{
        name=Datenmatrix,
        description={Die Darstellung der Features des Trainingsdatensatzes in tabelarischer Form. Jede Zeile beinhaltet die Features zu einer Probe.}
}

\newglossaryentry{Zielvektor}
{
        name=Zielvektor,
        description={Ein Vektor in einem Datensatz, der die zu vorhersagenden oder zu klassifizierenden Ausgaben enthält.}
}

\newglossaryentry{Labelvektor}
{
        name=Labelvektor,
        description={Ein Vektor, der die tatsächlichen Kategorien oder Ergebnisse für Trainingsdaten in einem überwachten Lernprozess enthält.}
}

\newglossaryentry{Featurevektor}
{
        name=Featurevektor,
        description={Ein Vektor, der alle Merkmale (Features) eines Datenelements repräsentiert, oft verwendet als Eingabe in maschinelle Lernmodelle.}
}

\newglossaryentry{Modellparameter}
{
        name=Modellparameter,
        description={Modellparameter sind die Parameter, welche ein Modell während des Trainingsprozesses selbstständig erlernt.}
}


\newglossaryentry{Hyperparameter}
{
        name=Hyperparameter,
        description={Hyperparameter sind die Parameter, welche ein Nutzer vor dem Modelltraining einstellen muss. Das Modell nimmt auf die Hyperparameter währen des Trainings keinen Einfluss. Über die Hyperparameter kann der Trainingsprozess beeinflusst werden.}
}

\newglossaryentry{Modelltraining}
{
        name=Modelltraining,
        description={Der Prozess, in dem ein maschinelles Lernmodell aus den Features lernt und die Modellparameter anpasst, um eine Zielfunktion zu minimieren oder zu maximieren.}
}

\newglossaryentrywithacronym
{MSE}
{Mittlerer quadratischer Fehler}
{Ein Maß für die durchschnittliche quadratische Differenz zwischen den vorhergesagten und den tatsächlichen Werten eines maschinellen Lernmodells. Häufig verwendet als Verlustfunktion, um Modelle zu trainieren.}

\newglossaryentry{Verlustfunktion}
{
        name=Verlustfunktion,
        description={Eine Funktion, die den Fehler eines Modells quantifiziert; das Ziel des Trainings ist es, diese Funktion zu minimieren.}
}

\newglossaryentry{Zielfunktion}
{
        name=Zielfunktion,
        description={Eine Funktion, für welche ein maschinelles Lernmodell im Training ein Optimum sucht, um die Modellparameter zu ermitteln.}
}

\newglossaryentry{Gradientenverfahren}
{
        name=Gradientenverfahren,
        description={Ein Optimierungsverfahren zum Lösen der Zielfunktion eines maschinellen Lernmodells durch schrittweise Anpassung der Modellparameter in Richtung des steilsten Abstiegs. Es wird nach dem Minimum gesucht.}
}

\newglossaryentry{Klassifikation}
{
        name=Klassifikation,
        description={Ein maschinelles Lernverfahren, bei dem das Modell einer unbekannten Probe eine Klasse zuteilt. Das geschieht in Form eines Labels.}
}


\newglossaryentry{Mehrklassen Klassifizierung}
{
        name=Mehrklassen Klassifizierung,
        description={Ein maschinelles Lernverfahren der Klassifikation, bei dem das Modell zwischen mehr als zwei Klassen unterscheiden können muss. Viele Modelle sind nur auf die binäre Klassifikation ausgelegt.}
}

\newglossaryentry{Label}
{
        name=Label,
        description={Ein Label ist eine Bezeichnung, die einer Probe in einem überwachten Lernprozess zugeordnet wird. Sie dient im Lernprozess dem Modell als Referenz, um das Schätzen einer Klasse oder eines Wert zu erlernen.}
}

\newglossaryentry{überwachtes Lernen}
{
        name=überwachtes Lernen,
        description={Ein Lernverfahren, bei dem das Modell aus einem Datensatz lernt, der sowohl Features als auch dazu gehörige Labels enthält.}
}

\newglossaryentry{unüberwachtes Lernen}
{
        name=unüberwachtes Lernen,
        description={Ein Lernverfahren,  bei dem das Modell aus einem Datensatz lernt, der nur Features enthält. Labels werden nicht verwendet. Es wird versucht, Muster zu finden, ohne dass Referenzen gegeben werden, wie diese Muster ausehen.}
}

\newglossaryentry{Deep Learning}
{
        name=Deep Learning,
        description={Ein Teilbereich des maschinellen Lernens, der Netzwerke mit vielen Schichten verwendet, um komplexe Muster in Daten zu erkennen. Es nutzt einfachen Lern-Algorithmen, um komplexe Modelle aufzubauen.}
}

\newglossaryentry{Generalisierung}
{
        name=Generalisierung,
        description={Die Fähigkeit eines maschinellen Lernmodells, auf neuen, unbekannten Daten gut zu performen, nachdem es auf einem Trainingsdatensatz gelernt hat.}
}

\newglossaryentry{Overfitting}
{
        name=Overfitting,
        description={Ein Modell lernt beim Training die Trainingsdaten auswendig. Es ist dadurch nicht in der Lage zu generalisieren.}
}

\newglossaryentry{Underfitting}
{
        name=Underfitting,
        description={Ein Modell lernt beim Training die Trainingsdaten unzureichend. Der Sachverhalt wird nicht gut vom Modell erfasst. Es ist dadurch nicht in der Lage zu generalisieren. Auch die Performance auf den Trainingsdaten ist schlecht.}
}

\newglossaryentry{Regularisierung}
{
        name=Regularisierung,
        description={Eine Technik zur Vermeidung von Overfitting, indem die Komplexität des Modells eingeschränkt wird. Regularisierung kann über Hyperparameter eingestellt werden.}
}

\newglossaryentry{Bias}
{
        name=Bias,
        description={Bias ist eine Voreingenommenheit des Modells. Unerkannt kann Bias eine Verzerrung der Schätzung des Modells verursachen, worunter die Performance leidet. Gezielt eingesetzt kann Bias helfen eine spezifische Aufgabe besser zu modellieren.}
}

\newglossaryentry{Leakage}
{
        name=Leakage,
        description={Das Auftreten von Informationen im Trainingsdatensatz, die bei der Modellerstellung nicht zur Verfügung stehen sollten, was zu einer überoptimistischen Schätzung der Modellleistung führt. Befinden sich Daten des Testdatensatzes auch im Trainingsdatensatz, tritt Leakage auf.}
}

\newglossaryentry{Machine Learning Workflow}
{
        name=Machine Learning Workflow,
        description={Der systematische Prozess zur Entwicklung, Evaluierung und Anwendung von Algorithmen des maschinellen Lernens.}
}

\newglossaryentry{Wrapper Methoden}
{
        name=Wrapper Methode,
        description={Methode zur Feature-Auswahl im Machine Learning Workflow. Die Features werden multivariat und modellabhängig beurteilt. Der Auswahlprozess ist iterativ. Verschieden Feature-Kombinationen werden getestet und anhand der Modellperformance ausgewählt. }
}

\newglossaryentry{Filter Methoden}
{
        name=Filter Methoden,
        description={Methode zur Feature-Auswahl im Machine Learning Workflow. Die Features werden univariat und modellunabhängig mit statistischen Methoden beurteilt. }
}

\newglossaryentry{Embedded Methoden}
{
        name=Embedded Methoden,
        description={Methode zur Feature-Auswahl im Machine Learning Workflow. Die Features werden multivariat und modellabhängig beurteilt. Die Auswahl findet wärend des Trainings statt. Embedded Methoden sind im Modell selbst implementiert.}
}

\newglossaryentry{Testdatensatz}
{
        name=Testdatensatz,
        description={Ein Datensatz, der zur Bewertung der Performance eines maschinellen Lernmodells verwendet wird, nachdem es mit einem Trainingsdatensatz trainiert wurde.}
}

\newglossaryentry{Trainingsdatensatz}
{
        name=Trainingsdatensatz,
        description={Ein Datensatz, der zum Trainieren eines maschinellen Lernmodells verwendet wird, um Muster in den Daten zu erlernen.}
}

\newglossaryentry{Accuracy}
{
        name=Accuracy,
        description={Eine Evaluationsmetrik für maschinelle Lernmodelle. Die Genauigkeit eines Modells, gemessen als Verhältnis der korrekt klassifizierten Proben zur der Gesamtanzahl aller Proben.}
}

\newglossaryentry{Konfusionsmatrix}
{
        name=Konfusionsmatrix,
        description={Eine Evaluationsmetrik für maschinelle Lernmodelle. Eine Konfusionsmatrix bietet Einblick in die Fehler die das Modell macht. Sie zeigt, welche Verwechslungen passieren. Sie stellt die Evaluation tabellarisch dar.}
}

\newglossaryentry{Cross-Validation}
{
        name=Cross-Validation,
        description={Eine Evaluationsmethode für maschinelle Lernmodelle, mit der kein zusätzlicher Validierungsdatensatz benötigt wird. Der Trainingsdatensatz wird in kleinere Teile aufgeteilt. Das Modell wird auf einem Teil des Datensatzes trainiert und auf einem anderen validiert. Dies wird mit allen Teilen wiederholt. Das mittlere Validierungsergbnis ist das Gesamtergebnis.}
}

\newglossaryentrywithacronym
{IMDB}
{In-Memory-Datenbank}
{Ein Datenbanksystem, das Daten im Arbeitsspeicher eines Computers speichert, um schnellere Zugriffszeiten im Vergleich zu datenträgerbasierten Datenbanken zu erreichen.}


%%%%%%%%%%%%%%%%%%%%%%%%%%%%%%%%%%%%%%%%%%%%%%%%%%%%%%%%%%%%%%%%%%%%
%------------------Software-----------------------------------------
%%%%%%%%%%%%%%%%%%%%%%%%%%%%%%%%%%%%%%%%%%%%%%%%%%%%%%%%%%%%%%%%%%%%

\newglossaryentry{Python}
{
        name=Python,
        description={Python ist eine höhere Programmiersprache. Sie arbeitet mit einem Codeinterpreter und ist objektorientiert. Python zeichnet sich durch eine klare und gut lesbare Syntax aus \cite{Pythondocumentation.20240309}.}
}

\newglossaryentry{Bibliothek}
{
        name=Bibliothek,
        description={In der Programmierung eine Sammlung von Unterprogrammen, die für die Entwicklung von Software verwendet werden können, um bestimmte Aufgaben zu erleichtern. Oft bieten Bibliotheken  spezialisierte Funktionalität. \cite{Burkov.2019}}
}

%%%%%%%%%%%%%%%%%%%%%%%%%%%%%%%%%%%%%%%%%%%%%%%%%%%%%%%%%%%%%%%%%%%%
%------------------Acronyme-----------------------------------------
%%%%%%%%%%%%%%%%%%%%%%%%%%%%%%%%%%%%%%%%%%%%%%%%%%%%%%%%%%%%%%%%%%%%

\newacronym{ID}{ID}{Identifikationsnummer}

\newacronym{SORT}{SORT}{Simple Online and Realtime Tracking}

\newacronym{OptiLiMa}{OptiLiMa}{Optimierung des Lichtmanagements für die Haltung von Mastputen}

\newacronym{CLEAR}{CLEAR}{Classification of Events, Activities and Relationships}

\newacronym{SVM}{SVM}{Support Vektor Machine}

\newacronym{LSTM}{LSTM}{Long Short Term Memory}

\newacronym{RNN}{RNN}{Rekurrentes neuronales Netzwerk}

\newacronym{GRU}{GRU}{Gated recurrent Unit}

\newacronym{FIFO}{FIFO}{First In - First Out}


